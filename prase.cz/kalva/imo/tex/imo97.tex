\nopagenumbers
\noindent {\bf 38th IMO 1997}
\vskip 25pt
\noindent {\bf A1}. In the plane the points with integer coordinates are the vertices of unit squares. The squares are colored alternately black and white as on a chessboard. For any pair of positive integers $m$ and $n$, consider a right-angled triangle whose vertices have integer coordinates and whose legs, of lengths $m$ and $n$, lie along the edges of the squares. Let $S_1$ be the total area of the black part of the triangle, and $S_2$ be the total area of the white part. Let $f(m, n) = |S_1 - S_2|$.

(a) Calculate $f(m,n)$ for all positive integers $m$ and $n$ which are either both even or both odd.

(b) Prove that $f(m,n) \le$ max$(m,n)/2$ for all $m$, $n$.

(c) Show that there is no constant $C$ such that $f(m,n) < C$ for all $m$, $n$.
\vskip 12pt
\noindent {\bf A2}. $\angle A$ is the smallest angle in the triangle $ABC$. The points $B$ and $C$ divide the circumcircle of the triangle into two arcs. Let $U$ be an interior point of the arc between $B$ and $C$ which does not contain $A$. The perpendicular bisectors of $AB$ and $AC$ meet the line $AU$ at $V$ and $W$, respectively. The lines $BV$ and $CW$ meet at $T$. Show that $AU = TB + TC$.
\vskip 12pt
\noindent {\bf A3}. Let $x_1, x_2, \ldots , x_n$ be real numbers satisfying $|x_1 + x_2 + \cdots + x_n| = 1$ and $|x_i| \le (n+1)/2$ for all $i$. Show that there exists a permutation $y_i$ of $x_i$ such that $|y_1 + 2 y_2 + \cdots + n y_n| \le (n+1)/2$.
\vskip 12pt
\noindent {\bf B1}. An $n \times n$ matrix whose entries come from the set $S = \{1, 2, \ldots , 2n-1\}$ is called a silver matrix if, for each $i = 1, 2, \ldots , n$, the $i\/$th row and the $i\/$th column together contain all elements of $S$. Show that:

(a) there is no silver matrix for $n = 1997$;

(b) silver matrices exist for infinitely many values of $n$.
\vskip 12pt
\noindent {\bf B2}. Find all pairs $(a,b)$ of positive integers that satisfy $a^{b^2} = b^a$.
\vskip 12pt
\noindent {\bf B3}. For each positive integer $n$, let $f(n)$ denote the number of ways of representing $n$ as a sum of powers of $2$ with non-negative integer exponents. Representations which differ only in the ordering of their summands are considered to be the same. For example, $f(4) = 4$, because $4$ can be represented as $4, 2+2, 2+1+1$ or $1+1+1+1$. Prove that for any integer $n \ge 3, 2^{n^2/4} < f(2^n) < 2^{n^2/2}$.
\vskip 20pt
\noindent \copyright John Scholes

\noindent jscholes@kalva.demon.co.uk

\noindent 21 August 2003

\bye
