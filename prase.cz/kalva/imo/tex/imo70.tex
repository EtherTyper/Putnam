\nopagenumbers
\noindent {\bf 12th IMO 1970}
\vskip 25pt
\noindent {\bf A1}. $M$ is any point on the side $AB$ of the triangle $ABC$. $r,r_1,r_2$ are the radii of the circles inscribed in $ABC,AMC,BMC$. $q$ is the radius of the circle on the opposite side of $AB$ to $C$, touching the three sides of $AB$ and the extensions of $CA$ and $CB$. Similarly, $q_1$ and $q_2$. Prove that $r_1r_2q=rq_1q_2$.
\vskip 12pt
\noindent {\bf A2}. We have $0\le x_i<b$ for $i=0,1,\ldots,n$ and $x_n>0,x_{n-1}>0$. If $a>b$, and $x_nx_{n-1}\ldots x_0$ represents the number $A$ base $a$ and $B$ base $b$, whilst $x_{n-1}x_{n-2}\ldots x_0$ represents the number $A'$ base $a$ and $B'$ base $b$, prove that $A'B<AB'$.
\vskip 12pt
\noindent {\bf A3}. The real numbers $a_0,a_1,a_2,\ldots$ satisfy $1=a_0\le a_1\le a_2\le\ldots. b_1,b_2,b_3,\ldots$ are defined by $b_n=\sum_1^n{1-{a_{k-1}\over a_k}\over\sqrt a_k}$.

(a) Prove that $0\le b_n<2$.

(b) Given $c$ satisfying $0\le c<2$, prove that we can find $a_n$ so that $b_n>c$ for all sufficiently large $n$.
\vskip 12pt
\noindent {\bf B1}. Find all positive integers $n$ such that the set $\{n,n+1,n+2,n+3,n+4,n+5\}$ can be partitioned into two subsets so that the product of the numbers in each subset is equal.
\vskip 12pt
\noindent {\bf B2}. In the tetrahedron $ABCD,\angle BDC=90^o$ and the foot of the perpendicular from $D$ to $ABC$ is the intersection of the altitudes of $ABC$. Prove that: $$(AB+BC+CA)^2\le6(AD^2+BD^2+CD^2).$$ When do we have equality?
\vskip 12pt
\noindent {\bf B3}. Given $100$ coplanar points, no three collinear, prove that at most $70\%$ of the triangles formed by the points have all angles acute.
\vskip 20pt
\noindent \copyright John Scholes

\noindent jscholes@kalva.demon.co.uk

\noindent 19 August 2003

\bye
