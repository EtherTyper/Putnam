\nopagenumbers
\noindent {\bf 16th IMO 1974}
\vskip 25pt
\noindent {\bf A1}. Three players play the following game. There are three cards each with a different positive integer. In each round, the cards are randomly dealt to the players and each receives the number of counters on his card. After two or more rounds, one player has received $20$, another $10$ and the third $9$ counters. In the last round the player with $10$ received the largest number of counters. Who received the middle number on the first round?
\vskip 12pt
\noindent {\bf A2}. Prove that there is a point $D$ on the side $AB$ of the triangle $ABC$, such that $CD$ is the geometric mean of $AD$ and $DB$ iff $\sin A\sin B\le\sin^2{C\over2}$.
\vskip 12pt
\noindent {\bf A3}. Prove that $\sum_0^n{2n+1\choose2k+1}2^{3k}$ is not divisible by $5$ for any non-negative integer $n$.
\vskip 12pt
\noindent {\bf B1}. An $8\times8$ chessboard is divided into $p$ disjoint rectangles (along the lines between the squares), so that each rectangle has the same number of white squares as black squares, and each rectangle has a different number of squares. Find the maximum possible value of $p$ and all possible sets of rectangle sizes.
\vskip 12pt
\noindent {\bf B2}. Determine all possible values of ${a\over a+b+d}+{b\over a+b+c}+{c\over b+c+d}+{d\over a+c+d}$ for positive reals $a,b,c,d$.
\vskip 12pt
\noindent {\bf B3}. Let $P(x)$ be a polynomial with integer coefficients of degree $d>0$. Let $n$ be the number of distinct integer roots to $P(x)=1$ or $-1$. Prove that $n\le d+2$.
\vskip 20pt
\noindent \copyright John Scholes

\noindent jscholes@kalva.demon.co.uk

\noindent 19 August 2003

\bye
