\nopagenumbers
\noindent {\bf 40th IMO 1999}
\vskip 25pt
\noindent {\bf A1}. Find all finite sets $S$ of at least three points in the plane such that for all distinct points $A, B$ in $S$, the perpendicular bisector of $AB$ is an axis of symmetry for S.
\vskip 12pt
\noindent {\bf A2}. Let $n \ge 2$ be a fixed integer. Find the smallest constant $C$ such that for all non-negative reals $x_1, \ldots , x_n$: $$\sum_{i<j} x_i x_j (x_i^2 + x_j^2) \le C (\sum_i x_i)^4.$$ Determine when equality occurs.
\vskip 12pt
\noindent {\bf A3}. Given an $n \times n$ square board, with $n$ even. Two distinct squares of the board are said to be adjacent if they share a common side, but a square is not adjacent to itself. Find the minimum number of squares that can be marked so that every square (marked or not) is adjacent to at least one marked square. 
\vskip 12pt
\noindent {\bf B1}. Find all pairs $(n, p)$ of positive integers, such that: $p$ is prime; $n \le 2p$; and $(p-1)^n + 1$ is divisible by $n^{p-1}$.
\vskip 12pt
\noindent {\bf B2}. The circles $C_1$ and $C_2$ lie inside the circle $C$, and are tangent to it at $M$ and $N$, respectively. $C_1$ passes through the center of $C_2$. The common chord of $C_1$ and $C_2$, when extended, meets $C$ at $A$ and $B$. The lines $MA$ and $MB$ meet $C_1$ again at $E$ and $F$. Prove that the line $EF$ is tangent to $C_2$.
\vskip 12pt
\noindent {\bf B3}. Determine all functions $f: R \to R$ such that $f(x - f(y)) = f(f(y)) + x f(y) + f(x) - 1$ for all $x, y$ in $R$. [$R$ is the reals.]
\vskip 20pt
\noindent \copyright John Scholes

\noindent jscholes@kalva.demon.co.uk

\noindent 19 August 2003

\bye
