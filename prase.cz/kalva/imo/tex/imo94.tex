\nopagenumbers
\noindent {\bf 35th IMO 1994}
\vskip 25pt
\noindent {\bf A1}. Let $m$ and $n$ be positive integers. Let $a_1, a_2, \ldots , a_m$ be distinct elements of $\{1, 2, \ldots , n\}$ such that whenever $a_i+a_j \le n$ for some $i,j$ (possibly the same) we have $a_i+a_j=a_k$ for some $k$. Prove that: $$(a_1+\ldots a_m)\ge {(n+1) \over 2}.$$
\vskip 12pt
\noindent {\bf A2}. $ABC$ is an isosceles triangle with $AB=AC$, $M$ is the midpoint of $BC$ and $O$ is the point on the line $AM$ such that $OB$ is perpendicular to $AB$. $Q$ is an arbitrary point on $BC$ different from $B$ and $C$. $E$ lies on the line $AB$ and $F$ lies on the line $AC$ such that $E, Q, F$ are distinct and collinear Prove that $OQ$ is perpendicular to $EF$ iff $QE=QF$.
\vskip 12pt
\noindent {\bf A3}.  For any positive integer $k$, let $f(k)$ be the number of elements in the set $\{k+1, k+2, \ldots, 2k\}$ which have exactly three $1$s when written in base $2$. Prove that for each positive integer $m$, there is at least one $k$ with $f(k)=m$, and determine all $m$ for which there is exactly one $k$.
\vskip 12pt
\noindent {\bf B1}. Determine all ordered pairs $(m, n)$ of positive integers for which ${n^3+1 \over mn-1}$ is an integer. 
\vskip 12pt
\noindent {\bf B2}. Let $S$ be the set of all real numbers greater than $-1$. Find all functions $f\colon S\to S$ such that $f\bigl(x+f(y)+xf(y)\bigr) = y+f(x) +yf(x)$ for all x, y, and ${f(x)\over x}$ is strictly increasing on each of the intervals $-1<x<0$ and $0<x$.
\vskip 12pt
\noindent {\bf B3}. Show that there exists a set $A$ of positive integers with the following property: for any infinite set $S$ of primes, there exist two positive integers $m\in A$ and $n\notin A$, each of which is a product of $k$ distinct elements of $S$ for some $k\ge 2$.
\vskip 20pt
\noindent \copyright John Scholes

\noindent jscholes@kalva.demon.co.uk

\noindent 25 August 2003

\bye
