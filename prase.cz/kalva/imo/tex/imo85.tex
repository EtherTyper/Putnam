\nopagenumbers
\noindent {\bf 26th IMO 1985}
\vskip 25pt
\noindent {\bf A1}. A circle has center on the side $AB$ of the cyclic quadrilateral $ABCD$. The other three sides are tangent to the circle. Prove that $AD+BC=AB$.
\vskip 12pt
\noindent {\bf A2}. Let $n$ and $k$ be relatively prime positive integers with $k<n$. Each number in the set $M=\{1,2,3,\ldots,n-1\}$ is colored either blue or white. For each $i$ in $M$, both $i$ and $n-i$ have the same color. For each $i\ne k$ in $M$ both $i$ and $|i-k|$ have the same color. Prove that all numbers in $M$ must have the same color.
\vskip 12pt
\noindent {\bf A3}. For any polynomial $P(x)=a_0+a_1x+\ldots+a_kx^k$ with integer coefficients, the number of odd coefficients is denoted by $o(P)$. For $i-0,1,2,\ldots$ let $Q_i(x)=(1+x)^i$. Prove that if $i_1,i_2,\ldots,i_n$ are integers satisfying $0\le i_1<i_2<\ldots<i_n$, then: $$o(Q_{i_1}+Q_{i_2}+\ldots+Q_{i_n})\ge o(Q_{i_1}).$$
\vskip 12pt
\noindent {\bf B1}. Given a set $M$ of $1985$ distinct positive integers, none of which has a prime divisor greater than $23$, prove that $M$ contains a subset of $4$ elements whose product is the $4$th power of an integer.
\vskip 12pt
\noindent {\bf B2}. A circle center $O$ passes through the vertices $A$ and $C$ of the triangle $ABC$ and intersects the segments $AB$ and $BC$ again at distinct points $K$ and $N$ respectively. The circumcircles of $ABC$ and $KBN$ intersect at exactly two distinct points $B$ and $M$. Prove that $\angle OMB$ is a right angle.
\vskip 12pt
\noindent {\bf B3}. For every real number $x_1$, construct the sequence $x_1,x_2,\ldots$ by setting: $$x_{n+1}=x_n(x_n+{1\over n}).$$ Prove that there exists exactly one value of $x_1$ which gives $0<x_n<x_{n+1}<1$ for all $n$.
\vskip 20pt
\noindent \copyright John Scholes

\noindent jscholes@kalva.demon.co.uk

\noindent 19 August 2003

\bye
