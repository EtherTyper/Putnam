\nopagenumbers
\noindent {\bf 20th IMO 1978}
\vskip 25pt
\noindent {\bf A1}. $m$ and $n$ are positive integers with $m<n$. The last three decimal digits of $1978^m$ are the same as the last three decimal digits of $1978^n$. Find $m$ and $n$ such that $m+n$ has the least possible value.
\vskip 12pt
\noindent {\bf A2}. $P$ is a point inside a sphere. Three mutually perpendicular rays from $P$ intersect the sphere at points $U,V$ and $W$. $Q$ denotes the vertex diagonally oppposite $P$ in the parallelepiped determined by $PU,PV,PW$. Find the locus of $Q$ for all possible sets of such rays from $P$.
\vskip 12pt
\noindent {\bf A3}. The set of all positive integers is the union of two disjoint subsets $\{f(1),f(2),f(3),\ldots\}$, $\{g(1),g(2),g(3),\ldots\}$, where $f(1)<f(2)<\ldots$, and $g(1)<g(2)<g(3)<\ldots$, and $g(n)=f(f(n))+1$ for $n=1,2,3,\ldots$. Determine $f(240)$.
\vskip 12pt
\noindent {\bf B1}. In the triangle $ABC, AB=AC$. A circle is tangent internally to the circumcircle of the triangle and also to $AB,AC$ at $P,Q$ respectively. Prove that the midpoint of $PQ$ is the center of the incircle of the triangle.
\vskip 12pt
\noindent {\bf B2}. $\{a_k\}$ is a sequence of distinct positive integers. Prove that for all positive integers $n, \sum_1^n{a_k\over k^2}\ge\sum_1^n{1\over k}$.
\vskip 12pt
\noindent {\bf B3}. An international society has its members from six different countries. The list of members has $1978$ names, numbered $1,2,\ldots,1978$. Prove that there is at least one member whose number is the sum of the numbers of two members from his own country, or twice the number of a member from his own country.
\vskip 20pt
\noindent \copyright John Scholes

\noindent jscholes@kalva.demon.co.uk

\noindent 19 August 2003

\bye
