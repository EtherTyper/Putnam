\nopagenumbers
\noindent {\bf 14th IMO 1972}
\vskip 25pt
\noindent {\bf A1}. Given any set of ten distinct numbers in the range $10,11,\ldots,99$, prove that we can always find two disjoint subsets with the same sum.
\vskip 12pt
\noindent {\bf A2}. Given $n>4$, prove that every cyclic quadrilateral can be dissected into $n$ cyclic quadrilaterals.
\vskip 12pt
\noindent {\bf A3}. Prove that $(2m)!(2n)!$ is a multiple of $m!n!(m+n)!$ for any non-negative integers $m$ and $n$.
\vskip 12pt
\noindent {\bf B1}. Find all positive real solutions to:

$(x_1^2-x_3x_5)(x_2^2-x_3x_5)\le0$

$(x_2^2-x_4x_1)(x_3^2-x_4x_1)\le0$

$(x_3^2-x_5x_2)(x_4^2-x_5x_2)\le0$

$(x_4^2-x_1x_3)(x_5^2-x_1x_3)\le0$

$(x_5^2-x_2x_4)(x_1^2-x_2x_4)\le0$
\vskip 12pt
\noindent {\bf B2}. $f$ and $g$ are real-valued functions defined on the real line. For all $x$ and $y, f(x+y)+f(x-y)=2f(x)g(y)$. $f$ is not identically zero and $|f(x)|\le1$ for all $x$. Prove that $|g(x)|\le1$ for all $x$.
\vskip 12pt
\noindent {\bf B3}. Given four distinct parallel planes, prove that there exists a regular tetrahedron with a vertex on each plane.
\vskip 20pt
\noindent \copyright John Scholes

\noindent jscholes@kalva.demon.co.uk

\noindent 19 August 2003

\bye
