\nopagenumbers
\noindent {\bf 41st IMO 2000}
\vskip 25pt
\noindent {\bf A1}. $AB$ is tangent to the circles $CAMN$ and $NMBD$. $M$ lies between $C$ and $D$ on the line $CD$, and $CD$ is parallel to $AB$. The chords $NA$ and $CM$ meet at $P$; the chords $NB$ and $MD$ meet at $Q$. The rays $CA$ and $DB$ meet at $E$. Prove that $PE = QE$.
\vskip 12pt
\noindent {\bf A2}. $A, B, C$ are positive reals with product $1$. Prove that $(A - 1 + {1 \over B})(B - 1 + {1 \over C})(C - 1 + {1 \over A}) \le 1$.
\vskip 12pt
\noindent {\bf A3}. $k$ is a positive real. $N$ is an integer greater than $1$. $N$ points are placed on a line, not all coincident. A {\it move} is carried out as follows. Pick any two points $A$ and $B$ which are not coincident. Suppose that $A$ lies to the right of $B$. Replace $B$ by another point $B'$ to the right of $A$ such that $AB' = k BA$. For what values of $k$ can we move the points arbitrarily far to the right by repeated moves?
\vskip 12pt
\noindent {\bf B1}. $100$ cards are numbered $1$ to $100$ (each card different) and placed in $3$ boxes (at least one card in each box). How many ways can this be done so that if two boxes are selected and a card is taken from each, then the knowledge of their sum alone is always sufficient to identify the third box?
\vskip 12pt
\noindent {\bf B2}. Can we find $N$ divisible by just $2000$ different primes, so that $N$ divides $2^N + 1$? [$N$ may be divisible by a prime power.]
\vskip 12pt
\noindent {\bf B3}. $A_1A_2A_3$ is an acute-angled triangle. The foot of the altitude from $A_i$ is $K_i$ and the incircle touches the side opposite $A_i$ at $L_i$. The line $K_1K_2$ is reflected in the line $L_1L_2$. Similarly, the line $K_2K_3$ is reflected in $L_2L_3$ and $K_3K_1$ is reflected in $L_3L_1$. Show that the three new lines form a triangle with vertices on the incircle.
\vskip 20pt
\noindent \copyright John Scholes

\noindent jscholes@kalva.demon.co.uk

\noindent 19 August 2003

\bye
