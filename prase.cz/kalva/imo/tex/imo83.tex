\nopagenumbers
\noindent {\bf 24th IMO 1983}
\vskip 25pt
\noindent {\bf A1}. Find all functions $f$ defined on the set of positive reals which take positive real values and satisfy: 

$f(xf(y))=yf(x)$ for all $x,y$; and $f(x)\to0$ as $x\to\infty$.
\vskip 12pt
\noindent {\bf A2}. Let $A$ be one of the two distinct points of intersection of two unequal coplanar circles $C_1$ and $C_2$ with centers $O_1$ and $O_2$ respectively. One of the common tangents to the circles touches $C_1$ at $P_1$ and $C_2$ at $P_2$, while the other touches $C_1$ at $Q_1$ and $C_2$ at $Q_2$. Let $M_1$ be the midpoint of $P_1Q_1$ and $M_2$ the midpoint of $P_2Q_2$. Prove that $\angle O_1AO_2=\angle M_1AM_2$.
\vskip 12pt
\noindent {\bf A3}. Let $a,b$ and $c$ be positive integers, no two of which have a common divisor greater than $1$. Show that $2abc-ab-bc-ca$ is the largest integer which cannot be expressed in the form $xbc+yca+zab$, where $x,y,z$ are non-negative integers.
\vskip 12pt
\noindent {\bf B1}. Let $ABC$ be an equilateral triangle and $E$ the set of all points contained in the three segments $AB,BC$ and $CA$ (including $A,B$ and $C$). Determine whether, for every partition of $E$ into two disjoint subsets, at least one of the two subsets contains the vertices of a right-angled triangle.
\vskip 12pt
\noindent {\bf B2}. Is it possible to choose $1983$ distinct positive integers, all less than or equal to $10^5$, no three of which are consecutive terms of an arithmetic progression?
\vskip 12pt
\noindent {\bf B3}. Let $a,b$ and $c$ be the lengths of the sides of a triangle. Prove that $$a^2b(a-b)+b^2c(b-c)+c^2a(c-a)\ge0.$$ Determine when equality occurs.
\vskip 20pt
\noindent \copyright John Scholes

\noindent jscholes@kalva.demon.co.uk

\noindent 19 August 2003

\bye
