\nopagenumbers
\noindent {\bf 9th IMO 1967}
\vskip 25pt
\noindent {\bf A1}. The parallelogram $ABCD$ has $AB=a,AD=1,\angle BAD=A$, and the triangle $ABD$ has all angles acute. Prove that circles radius $1$ and center $A,B,C,D$ cover the parallelogram iff $$a\le\cos A+\sqrt3\sin A.$$
\vskip 12pt
\noindent {\bf A2}. Prove that a tetrahedron with just one edge length greater than $1$ has volume at most $1\over8$.
\vskip 12pt
\noindent {\bf A3}. Let $k,m,n$ be natural numbers such that $m+k+1$ is a prime greater than $n+1$. Let $c_s=s(s+1)$. Prove that $$(c_{m+1}-c_k)(c_{m+2}-c_k)\ldots(c_{m+n}-c_k)$$ is divisible by the product $c_1c_2\ldots c_n$.
\vskip 12pt
\noindent {\bf B1}. $A_0B_0C_0$ and $A_1B_1C_1$ are acute-angled triangles. Construct the triangle $ABC$ with the largest possible area which is circumscribed about $A_0B_0C_0$ (so $BC$ contains $B_0, CA$ contains $B_0$, and $AB$ contains $C_0$) and similar to $A_1B_1C_1$.
\vskip 12pt
\noindent {\bf B2}. $a_1,\ldots,a_8$ are reals, not all zero. Let $c_n=a_1^n+a_2^n+\ldots+a_8^n$ for $n=1,2,3,\ldots$. Given that an infinite number of $c_n$ are zero, find all $n$ for which $c_n$ is zero.
\vskip 12pt
\noindent {\bf B3}. In a sports contest a total of $m$ medals were awarded over $n$ days. On the first day one medal and $1\over7$ of the remaining medals were awarded. On the second day two medals and $1\over7$ of the remaining medals were awarded, and so on. On the last day, the remaining $n$ medals were awarded. How many medals were awarded, and over how many days?
\vskip 20pt
\noindent \copyright John Scholes

\noindent jscholes@kalva.demon.co.uk

\noindent 19 August 2003

\bye
