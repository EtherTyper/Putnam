\nopagenumbers
\noindent {\bf 31st IMO 1990}
\vskip 25pt
\noindent {\bf A1}. Chords $AB$ and $CD$ of a circle intersect at a point $E$ inside the circle. Let $M$ be an interior point of the segment $EB$. The tangent at $E$ to the circle through $D,E,M$ intersects the lines $BC$ and $AC$ at $F$ and $G$ respectively. Find $EF\over EF$ in terms of $t={AM\over AB}$.
\vskip 12pt
\noindent {\bf A2}. Take $n\ge3$ and consider a set $E$ of $2n-1$ distinct points on a circle. Suppose that exactly $k$ of these points are to be colored black. Such a coloring is {\it good} if there is at least one pair of black points such that the interior of one of the arcs between them contains exactly $n$ points from $E$. Find the smallest value of $k$ so that every such coloring of $k$ points of $E$ is good.
\vskip 12pt
\noindent {\bf A3}. Determine all integers greater than $1$ such that $2^n+1\over n^2$ is an integer.
\vskip 12pt
\noindent {\bf B1}. Construct a function from the set of positive rational numbers into itself such that $f(xf(y))={f(x)\over y}$ for all $x,y$.
\vskip 12pt
\noindent {\bf B2}. Given an initial integer $n_0>1$, two players $A$ and $B$ choose integers $n_1,n_2,n_3,\ldots$ alternately according to the following rules:

Knowing $n_{2k}, A$ chooses any integer $n_{2k+1}$ such that $n_{2k}\le n_{2k+1}\le n_{2k}^2$.

Knowing $n_{2k+1},B$ chooses any integer $n_{2k+2}$ such that ${n_{2k+1}\over n_{2k+2}}=p^r$ for some prime $p$ and integer $r\ge 1$.

\noindent Player $A$ wins the game by choosing the number $1990$; player $B$ wins by choosing the number $1$. For which $n_0$ does

(a) $A$ have a winning strategy?

(b) $B$ have a winning strategy?

(c) Neither player have a winning strategy?
\vskip 12pt
\noindent {\bf B3}. Prove that there exists a convex $1990$-gon such that all its angles are equal and the lengths of the sides are the numbers $1^2,2^2,\ldots, 1990^2$ in some order.
\vskip 20pt
\noindent \copyright John Scholes

\noindent jscholes@kalva.demon.co.uk

\noindent 19 August 2003

\bye
