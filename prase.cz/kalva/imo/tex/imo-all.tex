\noindent{\bf International Mathematical Olympiads}
\vskip 40pt
\noindent {\bf 1st IMO 1959}
\vskip 25pt
\noindent {\bf A1}. Prove that $21n+4\over14n+3$ is irreducible for every natural number $n$.
\vskip 12pt
\noindent {\bf A2}. For what real values of $x$ is $\sqrt{x+\sqrt{2x-1}}+\sqrt{x-\sqrt{2x-1}}=A$ given $A=\sqrt2$, (b) $A=1$, (c) $A=2$, where only non-negative real numbers are allowed in square roots and the root always denotes the non-negative root?
\vskip 12pt
\noindent {\bf A3}. Let $a,b,c$ be real numbers. Given the equation for $\cos x$: $$a\cos^2x+b\cos x+c=0,$$ form a quadratic equation in $\cos{2x}$ whose roots are the same values of $x$. Compare the equations in $\cos x$ and $\cos{2x}$ for $a=4,b=2,c=-1$.
\vskip 12pt
\noindent {\bf B1}. Given the length $|AC|$, construct a triangle $ABC$ with $\angle ABC=90^o$, and the median $BM$ satisfying $BM^2=AB\cdot BC$.
\vskip 12pt
\noindent {\bf B2}. An arbitrary point $M$ is taken in the interior of the segment $AB$. Squares $AMCD$ and $MBEF$ are constructed on the same side of $AB$. The circles circumscribed about these squares, with centers $P$ and $Q$, intersect at $M$ and $N$. 

(a) prove that $AF$ and $BC$ intersect at $N$;

(b) prove that the lines $MN$ pass through a fixed point $S$ (independent of $M$);

(c) find the locus of the midpoints of the segments $PQ$ as $M$ varies.
\vskip 12pt
\noindent {\bf B3}. The planes $P$ and $Q$ are not parallel. The point $A$ lies in $P$ but not $Q$, and the point $C$ lies in $Q$ but not $P$. Construct points $B$ in $P$ and $D$ in $Q$ such that the quadrilateral $ABCD$ satisfies the following conditions: (1) it lies in a plane, (2) the vertices are in the order $A,B,C,D$, (3) it is an isosceles trapezoid with $AB$ parallel to $CD$ (meaning that $AD=BC$, but $AD$ is not parallel to $BC$ unless it is a square), and (4) a circle can be inscribed in $ABCD$ touching the sides.
\vskip 40pt
\noindent {\bf 2nd IMO 1960}
\vskip 25pt
\noindent {\bf A1}. Determine all three digit numbers $N$ which are divisible by $11$ and where $N\over11$ is equal to the sum of the squares of the digits of $N$.
\vskip 12pt
\noindent {\bf A2}. For what real values of $x$ does the following inequality hold: $${4x^2\over(1-\sqrt{1+2x})^2}<2x+9?$$
\vskip 12pt
\noindent {\bf A3}. In a given right triangle $ABC$, the hypoteneuse $BC$, length $a$, is divided into $n$ equal parts with $n$ an odd integer. The central part subtends an angle $\alpha$ at $A$. $h$ is the perpendicular distance from $A$ to $BC$. Prove that $$\tan\alpha={4nh\over an^2-a}.$$
\vskip 12pt
\noindent {\bf B1}. Construct a triangle $ABC$ given the lengths of the altitudes from $A$ and $B$ and the length of the median from $A$.
\vskip 12pt
\noindent {\bf B2}. The cube $ABCDA'B'C'D'$ has $A$ above $A',B$ above $B'$ and so on. $X$ is any point of the face diagonal $AC$ and $Y$ is any point of $B'D'$.

(a) find the locus of the midpoint of $XY$;

(b) find the locus of the point $Z$ which lies one-third of the way along $XY$, so that $ZY=2XZ$.
\vskip 12pt
\noindent {\bf B3}. A cone of revolution has an inscribed sphere tangent to the base of the cone (and to the sloping surface of the cone). A cylinder is circumscribed about the sphere  so that its base lies in the base of the cone. The volume of the cone is $V_1$ and the volume of the cylinder is $V_2$.

(a) Prove that $V_1\ne V_2$;

(b) Find the smallest possible value of $V_1\over V_2$. For this case construct the half angle of the cone.
\vskip 12pt
\noindent {\bf B4}. In the isosceles trapezoid $ABCD$ ($AB$ parallel to $DC$, and $BC=AD$), let $AB=a,CD=c$ and let the perpendicular distance from $A$ to $CD$ be $h$. Show how to construct all points $X$ on the axis of symmetry such that $\angle BXC=\angle AXD=90^o$. Find the distance of each such $X$ from $AB$ and from $CD$. What is the condition for such points to exist?
\vskip 40pt
\noindent {\bf 3rd IMO 1961}
\vskip 25pt
\noindent {\bf A1}. Solve the following equations for $x,y$ and $z$: $$x+y+z=a; x^2+y^2+z^2=b^2; xy=z^2.$$ What conditions must $a$ and $b$ satisfy for $x,y$ and $z$ to be distinct positive numbers?
\vskip 12pt
\noindent {\bf A2}. Let $a,b,c$ be the sides of a triangle and $A$ its area. Prove that: $$a^2+b^2+c^2\ge4\sqrt3A$$ When do we have equality?
\vskip 12pt
\noindent {\bf A3}. Solve the equation $\cos^nx-\sin^nx=1$, where $n$ is a natural number.
\vskip 12pt
\noindent {\bf B1}. $P$ is inside the triangle $ABC$. $PA$ intersects $BC$ in $D,PB$ intersects $AC$ in $E$, and $PC$ intersects $AB$ in $F$. Prove that at least one of ${AP\over PD}, {BP\over PE},{CP\over PF}$ does not exceed $2$, and at least one is not less than $2$.
\vskip 12pt
\noindent {\bf B2}. Construct the triangle $ABC$, given the lengths $AC=b,AB=c$ and the acute $\angle AMB=\alpha$, where $M$ is the midpoint of $BC$. Prove that the construction is possible iff $$b\tan{\alpha\over2}\le c<b.$$ When does equality hold?
\vskip 12pt
\noindent {\bf B3}. Given three non-collinear points $A,B,C$ and a plane $p$ not parallel to $ABC$ and such that $A,B,C$ are all on the same side of $p$. Take three arbitrary points $A',B',C'$ in $p$. Let $A'',B'',C''$ be the midpoints of $AA',BB',CC'$ respectively, and let $O$ be the centroid of $A'',B'',C''$. What is the locus of $O$ as $A',B',C'$ vary?
\vskip 40pt
\noindent {\bf 4th IMO 1962}
\vskip 25pt
\noindent {\bf A1}. Find the smallest natural number with $6$ as the last digit, such that if the final $6$ is moved to the front of the number it is multiplied by $4$.
\vskip 12pt
\noindent {\bf A2}. Find all real $x$ satisfying: $\sqrt{3-x}-\sqrt{x+1}>{1\over2}$.
\vskip 12pt
\noindent {\bf A3}. The cube $ABCDA'B'C'D'$ has upper face $ABCD$ and lower face $A'B'C'D'$ with $A$ directly above $A'$ and so on. The point $x$ moves at a constant speed along the perimeter of $ABCD$, and the point $Y$ moves at the same speed along the perimeter of $B'C'CB$. $X$ leaves $A$ towards $B$ at the same moment as $Y$ leaves $B'$ towards $C'$. What is the locus of the midpoint of $XY$?
\vskip 12pt
\noindent {\bf B1}. Find all real solutions to $\cos^2x+\cos^2{2x}+\cos^2{3x}=1$.
\vskip 12pt
\noindent {\bf B2}. Given three distinct points $A,B,C$ on a circle $K$, construct a point $D$ on $K$, such that a circle can be inscribed in $ABCD$.
\vskip 12pt
\noindent {\bf B3}. The radius of the circumcircle of an isosceles triangle is $R$ and the radius of its inscribed circle is $r$. Prove that the distance between the two centers is $\sqrt{R(R-2r)}$.
\vskip 12pt
\noindent {\bf B4}. Prove that a regular tetrahedron has five distinct spheres each tangent to its six extended edges. Conversely, prove that if a tetrahedron has five such spheres then it is regular.
\vskip 40pt
\noindent {\bf 5th IMO 1963}
\vskip 25pt
\noindent {\bf A1}. For which real values of $p$ does the equation $$\sqrt{x^2-p}+2\sqrt{x^2-1}=x$$ have real roots? What are the roots?
\vskip 12pt
\noindent {\bf A2}. Given a point $A$ and a segment $BC$, determine the locus of all points $P$ in space for which $\angle APX=90^o$ for some $X$ on the segment $BC$.
\vskip 12pt
\noindent {\bf A3}. An $n$-gon has all angles equal and the lengths of consecutive sides satisfy $a_1\ge a_2\ge\ldots\ge a_n$. Prove that all the sides are equal.
\vskip 12pt
\noindent {\bf B1}. Find all solutions $x_1,\ldots,x_5$ to the five equations $x_i+x_{i+2}=yx_{i+1}$ for $i=1,\ldots,5$, where subscripts are reduced by $5$ if necessary.
\vskip 12pt
\noindent {\bf B2}. Prove that $\cos{\pi\over7}-\cos{2\pi\over7}+\cos{3\pi\over7}={1\over2}$.
\vskip 12pt
\noindent {\bf B3}. Five students $A,B,C,D,E$ were placed $1$ to $5$ in a contest with no ties. One prediction was that the result would be the order $A,B,C,D,E$. But no student finished in the position predicted and no two students predicted to finish consecutively did so. For example, the outcome for $C$ and $D$ was not $1,2$ (respectively), or $2,3$, or $3,4$ or $4,5$. Another prediction was the order $D,A,E,C,B$. Exactly two students finished in the places predicted and two disjoint pairs predicted to finish consecutively did so. Determine the outcome.
\vskip 40pt
\noindent {\bf 6th IMO 1964}
\vskip 25pt
\noindent {\bf A1}. (a) Find all natural numbers $n$ for which $7$ divides $2^n-1$.

(b) Prove that there is no natural number $n$ for which $7$ divides $2^n+1$.
\vskip 12pt
\noindent {\bf A2}. Suppose that $a,b,c$ are the sides of a triangle. Prove that: $$a^2(b+c-a)+b^2(c+a-b)+c^2(a+b-c)\le3abc.$$
\vskip 12pt
\noindent {\bf A3}. Triangle $ABC$ has sides $a,b,c$. Tangents to the inscribed circle are constructed parallel to the sides. Each tangent forms a triangle with the other two sides of the triangle and a circle is inscribed in each of these triangles. Find the total area of all four inscribed circles.
\vskip 12pt
\noindent {\bf B1}. Each pair from $17$ people exchange letters on one of three topics. Prove that there are at least $3$ people who write to each other on the same topic. [In other words, if we color the edges of the complete graph $K_17$ with three colors, then we can find a triangle all the same color.]
\vskip 12pt
\noindent {\bf B2}. $5$ points in a plane are situated so that no two of the lines joining a pair of points are coincident, parallel or perpendicular. Through each point lines are drawn perpendicular to each of the lines through two of the other $4$ points. Determine the maximum number of intersections these perpendiculars can have.
\vskip 12pt
\noindent {\bf B3}. $ABCD$ is a tetrahedron and $D_0$ is the centroid of $ABC$. Lines parallel to $DD_0$ are drawn through $A,B$ and $C$ and meet the planes $BCD,CAD$ and $ABD$ in $A_0,B_0$ and $C_0$ respectively. Prove that the volume of $ABCD$ is one-third of the volume of $A_0B_0C_0D_0$. Is the result true if $D_0$ is an arbitrary point inside $ABC$?
\vskip 40pt
\noindent {\bf 7th IMO 1965}
\vskip 25pt
\noindent {\bf A1}. Find all $x$ in the interval $[0,2\pi]$ which satisfy: $$2\cos x\le|\sqrt{1+\sin{2x}}-\sqrt{1-\sin{2x}}|\le\sqrt2.$$
\vskip 12pt
\noindent {\bf A2}. The coefficients $a_{ij}$ of the following equations

$a_{11}x_1+a_{12}x_2+a_{13}x_3=0$

$a_{21}x_1+a_{22}x_2+a_{23}x_3=0$

$a_{31}x_1+a_{32}x_2+a_{33}x_3=0$

\noindent satisfy the following: (a) $a_{11},a_{22},a_{33}$ are positive, (b) other $a_{ij}$ are negative, (c) the sum of the coefficients in each equation is positive. Prove that the only solution is $x_1=x_2=x_3=0$.
\vskip 12pt
\noindent {\bf A3}. The tetrahedron $ABCD$ is divided into two parts by a plane parallel to $AB$ and $CD$. The distance of the plane from $AB$ is $k$ times its distance from $CD$. Find the ratio of the volumes of the two parts.
\vskip 12pt
\noindent {\bf B1}. Find all sets of four real numbers such that the sum of any one and the product of the other three is $2$.
\vskip 12pt
\noindent {\bf B2}. The triangle $OAB$ has $\angle O$ acute. $M$ is an arbitrary point on $AB$. $P$ and $Q$ are the feet of the perpendiculars from $M$ to $OA$ and $OB$ respectively. What is the locus of $H$, the orthocenter of the triangle $OPQ$ (the point where its altitudes meet)? What is the locus if $M$ is allowed to vary over the interior of $OAB$?
\vskip 12pt
\noindent {\bf B3}. Given $n>2$ points in the plane, prove that at most $n$ pairs of points are the maximum distance apart (of any two points in the set).
\vskip 40pt
\noindent {\bf 8th IMO 1966}
\vskip 25pt
\noindent {\bf A1}. Problems $A,B$ and $C$ were posed in a mathematical contest. $25$ competitors solved at least one of the three. Amongst those who did not solve $A$, twice as many solved $B$ as $C$. The number solving only $A$ was one more than the number solving $A$ and at least one other. The number solving just $A$ equalled the number solving just $B$ plus the number solving just $C$. How many solved just $B$?
\vskip 12pt
\noindent {\bf A2}. Prove that if $BC+AC=\tan{C\over2}(BC\tan A+AC\tan B)$, then the triangle $ABC$ is isosceles.
\vskip 12pt
\noindent {\bf A3}. Prove that a point in space has the smallest sum of the distances to the vertices of a regular tetrahedron iff it is the center of the tetrahedron.
\vskip 12pt
\noindent {\bf B1}. Prove that ${1\over\sin{2x}}+{1\over\sin{4x}}+\ldots+{1\over\sin{2^nx}}=\cot x-\cot{2^nx}$ for any natural number $n$ and any real $x$ (with $\sin{2^nx}$ non-zero).
\vskip 12pt
\noindent {\bf B2}. Solve the equations $$|a_i-a_1|x_1+|a_i-a_2|x_2+|a_i-a_3|x_3+|a_i-a_4|x_4=1,i=1,2,3,4,$$ where $a_i$ are distinct reals.
\vskip 12pt
\noindent {\bf B3}. Take any points $K,L,M$ on the sides $BC,CA,AB$ of the triangle $ABC$. Prove that at least one of the triangles $AML,BKM,CLK$ has area $\le{1\over4}$ area $ABC$.
\vskip 40pt
\noindent {\bf 9th IMO 1967}
\vskip 25pt
\noindent {\bf A1}. The parallelogram $ABCD$ has $AB=a,AD=1,\angle BAD=A$, and the triangle $ABD$ has all angles acute. Prove that circles radius $1$ and center $A,B,C,D$ cover the parallelogram iff $$a\le\cos A+\sqrt3\sin A.$$
\vskip 12pt
\noindent {\bf A2}. Prove that a tetrahedron with just one edge length greater than $1$ has volume at most $1\over8$.
\vskip 12pt
\noindent {\bf A3}. Let $k,m,n$ be natural numbers such that $m+k+1$ is a prime greater than $n+1$. Let $c_s=s(s+1)$. Prove that $$(c_{m+1}-c_k)(c_{m+2}-c_k)\ldots(c_{m+n}-c_k)$$ is divisible by the product $c_1c_2\ldots c_n$.
\vskip 12pt
\noindent {\bf B1}. $A_0B_0C_0$ and $A_1B_1C_1$ are acute-angled triangles. Construct the triangle $ABC$ with the largest possible area which is circumscribed about $A_0B_0C_0$ (so $BC$ contains $B_0, CA$ contains $B_0$, and $AB$ contains $C_0$) and similar to $A_1B_1C_1$.
\vskip 12pt
\noindent {\bf B2}. $a_1,\ldots,a_8$ are reals, not all zero. Let $c_n=a_1^n+a_2^n+\ldots+a_8^n$ for $n=1,2,3,\ldots$. Given that an infinite number of $c_n$ are zero, find all $n$ for which $c_n$ is zero.
\vskip 12pt
\noindent {\bf B3}. In a sports contest a total of $m$ medals were awarded over $n$ days. On the first day one medal and $1\over7$ of the remaining medals were awarded. On the second day two medals and $1\over7$ of the remaining medals were awarded, and so on. On the last day, the remaining $n$ medals were awarded. How many medals were awarded, and over how many days?
\vskip 40pt
\noindent {\bf 10th IMO 1968}
\vskip 25pt
\noindent {\bf A1}. Find all triangles whose side lengths are consecutive integers, and one of whose angles is twice another.
\vskip 12pt
\noindent {\bf A2}. Find all natural numbers $n$ the product of whose decimal digits is $n^2-10n-22$.
\vskip 12pt
\noindent {\bf A3}. $a,b,c$ are real with $a$ non-zero. $x_1,x_2,\ldots,x_n$ satisfy the $n$ equations:

$ax_i^2+bx_i+c=x_{i+1}$, for $1\le i<n$

$ax_n^2+bx_n+c=x_1$

\noindent Prove that the system has zero, $1$ or $>1$ real solutions according as $(b-1)^2-4ac$ is $<0,=0$, or $>0$.
\vskip 12pt
\noindent {\bf B1}. Prove that every tetrahedron has a vertex whose three edges have the right lengths to form a triangle.
\vskip 12pt
\noindent {\bf B2}. Let $f$ be a real-valued function defined for all real numbers, such that for some $a>0$ we have 

$f(x+a)={1\over2}+\sqrt{f(x)-f(x)^2}$ for all $x$.

\noindent Prove that $f$ is periodic, and give an example of such a non-constant $f$ for $a=1$.
\vskip 12pt
\noindent {\bf B3}. For every natural number $n$ evaluate the sum $$\Bigl[{n+1\over2}\Bigr]+\Bigl[{n+2\over4}\Bigr]+\Bigl[{n+4\over8}\Bigr]+\ldots+\Bigl[{n+2^k\over2^{k+1}}\Bigr]+\ldots,$$ where $[x]$ denotes the greatest integer $\le x$.
\vskip 40pt
\noindent {\bf 11th IMO 1969}
\vskip 25pt
\noindent {\bf A1}. Prove that there are infinitely many positive integers $m$, such that $n^4+m$ is not prime for any positive integer $n$.
\vskip 12pt
\noindent {\bf A2}. Let $f(x)=\cos(a_1+x)+{1\over2}\cos(a_2+x)+{1\over4}\cos(a_3+x)+\ldots+{1\over2^{n-1}}\cos(a_n+x)$, where $a_i$ are real constants and $x$ is a real variable. If $f(x_1)=f(x_2)=0$, prove that $x_1-x_2$ is a multiple of $\pi$.
\vskip 12pt
\noindent {\bf A3}. For each of $k=1,2,3,4,5$ find necessary and sufficient conditions on $a>0$ such that there exists a tetrahedron with $k$ edges length $a$ and the remainder length $1$.
\vskip 12pt
\noindent {\bf B1}. $C$ is a point on the semicircle diameter $AB$, between $A$ and $B$. $D$ is the foot of the perpendicular from $C$ to $AB$. The circle $K_1$ is the incircle of $ABC$, the circle $K_2$ touches $CD,DA$ and the semicircle, the circle $K_3$ touches $CD,DB$ and the semicircle. Prove that $K_1,K_2$ and $K_3$ have another common tangent apart from $AB$.
\vskip 12pt
\noindent {\bf B2}. Given $n>4$ points in the plane, no three collinear. Prove that there are at least $(n-3)(n-4)\over2$ convex quadrilaterals with vertices amongst the $n$ points.
\vskip 12pt
\noindent {\bf B3}. Given real numbers $x_1,x_2,y_1,y_2,z_1,z_2$ satisfying $x_1>0,x_2>0,x_1y_1>z_1^2$, and $x_2y_2>z_2^2$, prove that: $${8\over(x_1+x_2)(y_1+y_2)-(z_1+z_2)^2}\le{1\over x_1y_1-z_1^2}+{1\over x_2y_2-z_2^2}.$$ Give necessary and sufficient conditions for equality.
\vskip 40pt
\noindent {\bf 12th IMO 1970}
\vskip 25pt
\noindent {\bf A1}. $M$ is any point on the side $AB$ of the triangle $ABC$. $r,r_1,r_2$ are the radii of the circles inscribed in $ABC,AMC,BMC$. $q$ is the radius of the circle on the opposite side of $AB$ to $C$, touching the three sides of $AB$ and the extensions of $CA$ and $CB$. Similarly, $q_1$ and $q_2$. Prove that $r_1r_2q=rq_1q_2$.
\vskip 12pt
\noindent {\bf A2}. We have $0\le x_i<b$ for $i=0,1,\ldots,n$ and $x_n>0,x_{n-1}>0$. If $a>b$, and $x_nx_{n-1}\ldots x_0$ represents the number $A$ base $a$ and $B$ base $b$, whilst $x_{n-1}x_{n-2}\ldots x_0$ represents the number $A'$ base $a$ and $B'$ base $b$, prove that $A'B<AB'$.
\vskip 12pt
\noindent {\bf A3}. The real numbers $a_0,a_1,a_2,\ldots$ satisfy $1=a_0\le a_1\le a_2\le\ldots. b_1,b_2,b_3,\ldots$ are defined by $b_n=\sum_1^n{1-{a_{k-1}\over a_k}\over\sqrt a_k}$.

(a) Prove that $0\le b_n<2$.

(b) Given $c$ satisfying $0\le c<2$, prove that we can find $a_n$ so that $b_n>c$ for all sufficiently large $n$.
\vskip 12pt
\noindent {\bf B1}. Find all positive integers $n$ such that the set $\{n,n+1,n+2,n+3,n+4,n+5\}$ can be partitioned into two subsets so that the product of the numbers in each subset is equal.
\vskip 12pt
\noindent {\bf B2}. In the tetrahedron $ABCD,\angle BDC=90^o$ and the foot of the perpendicular from $D$ to $ABC$ is the intersection of the altitudes of $ABC$. Prove that: $$(AB+BC+CA)^2\le6(AD^2+BD^2+CD^2).$$ When do we have equality?
\vskip 12pt
\noindent {\bf B3}. Given $100$ coplanar points, no three collinear, prove that at most $70\%$ of the triangles formed by the points have all angles acute.
\vskip 40pt
\noindent {\bf 13th IMO 1971}
\vskip 25pt
\noindent {\bf A1}. Let $E_n=(a_1-a_2)(a_1-a_3)\ldots(a_1-a_n)+(a_2-a_1)(a_2-a_3)\ldots(a_2-a_n)+\ldots+(a_n-a_1)(a_n-a_2)\ldots(a_n-a_{n-1})$. Let $S_n$ be the proposition that $E_n\ge0$ for all real $a_i$. Prove that $S_n$ is true for $n=3$ and $5$, but for no other $n>2$.
\vskip 12pt
\noindent {\bf A2}. Let $P_1$ be a convex polyhedron with vertices $A_1,A_2,\ldots,A_9$. Let $P_i$ be the polyhedron obtained from $P_1$ by a translation that moves $A_1$ to $A_i$. Prove that at least two of the polyhedra $P_1,P_2,\ldots,P_9$ have an interior point in common.
\vskip 12pt
\noindent {\bf A3}. Prove that we can find an infinite set of positive integers of the from $2^n-3$ (where $n$ is a positive integer) every pair of which are relatively prime.
\vskip 12pt
\noindent {\bf B1}. All faces of the tetrahedron $ABCD$ are acute-angled. Take a point $X$ in the interior of the segment $AB$, and similarly $Y$ in $BC, Z$ in $CD$ and $T$ in $AD$.

(a) If $\angle DAB+\angle BCD\ne\angle CDA+\angle ABC$, then prove none of the closed paths $XYZTX$ has minimal length;

(b) If $\angle DAB+\angle BCD=\angle CDA+\angle ABC$, then there are infinitely many shortest paths $XYZTX$, each with length $2AC\sin k$, where $2k=\angle BAC+\angle CAD+\angle DAB$.
\vskip 12pt
\noindent {\bf B2}. Prove that for every positive integer $m$ we can find a finite set $S$ of points in the plane, such that given any point $A$ of $S$, there are exactly $m$ points in $S$ at unit distance from $A$.
\vskip 12pt
\noindent {\bf B3}. Let $A=(a_{ij})$, where $i,j=1,2,\ldots,n$, be a square matrix with all $a_{ij}$ non-negative integers. For each $i,j$ such that $a_{ij}=0$, the sum of the elements in the $i$th row and the $j$th column is at least $n$. Prove that the sum of all the elements in the matrix is at least $n^2\over2$.
\vskip 40pt
\noindent {\bf 14th IMO 1972}
\vskip 25pt
\noindent {\bf A1}. Given any set of ten distinct numbers in the range $10,11,\ldots,99$, prove that we can always find two disjoint subsets with the same sum.
\vskip 12pt
\noindent {\bf A2}. Given $n>4$, prove that every cyclic quadrilateral can be dissected into $n$ cyclic quadrilaterals.
\vskip 12pt
\noindent {\bf A3}. Prove that $(2m)!(2n)!$ is a multiple of $m!n!(m+n)!$ for any non-negative integers $m$ and $n$.
\vskip 12pt
\noindent {\bf B1}. Find all positive real solutions to:

$(x_1^2-x_3x_5)(x_2^2-x_3x_5)\le0$

$(x_2^2-x_4x_1)(x_3^2-x_4x_1)\le0$

$(x_3^2-x_5x_2)(x_4^2-x_5x_2)\le0$

$(x_4^2-x_1x_3)(x_5^2-x_1x_3)\le0$

$(x_5^2-x_2x_4)(x_1^2-x_2x_4)\le0$
\vskip 12pt
\noindent {\bf B2}. $f$ and $g$ are real-valued functions defined on the real line. For all $x$ and $y, f(x+y)+f(x-y)=2f(x)g(y)$. $f$ is not identically zero and $|f(x)|\le1$ for all $x$. Prove that $|g(x)|\le1$ for all $x$.
\vskip 12pt
\noindent {\bf B3}. Given four distinct parallel planes, prove that there exists a regular tetrahedron with a vertex on each plane.
\vskip 40pt
\noindent {\bf 15th IMO 1973}
\vskip 25pt
\noindent {\bf A1}. $OP_1,OP_2,\ldots,OP_{2n+1}$ are unit vectors in a plane. $P_1,P_2,\ldots,P_{2n+1}$ all lie on the same side of a line through $O$. Prove that $|OP_1+\ldots+OP_{2n+1}|\ge1$.
\vskip 12pt
\noindent {\bf A2}. Can we find a finite set of non-coplanar points, such that given any two points, $A$ and $B$, there are two others $C$ and $D$, with the lines $AB$ and $CD$ parallel and distinct?
\vskip 12pt
\noindent {\bf A3}. $a$ and $b$ are real numbers for which the equation $x^4+ax^3+bx^2+ax+1=0$ has at least one real solution. Find the least possible value of $a^2+b^2$.
\vskip 12pt
\noindent {\bf B1}. A soldier needs to sweep a region with the shape of an equilateral triangle for mines. The detector has an effective radius equal to half the altitude of the triangle. He starts at a vertex of the triangle. What path should he follow in order to travel the least distance and still sweep the whole region?
\vskip 12pt
\noindent {\bf B2}. $G$ is a set of non-constant functions $f$. Each $f$ is defined on the real line and has the form $f(x)=ax+b$ for some real $a,b$. If $f$ and $g$ are in $G$, then so is $fg$, where $fg$ is defined by $fg(x)=f(g(x))$. If $f$ is in $G$, then so is the inverse $f^{-1}$. If $f(x)=ax+b$, then $f^{-1}(x)={x-b\over a}$. Every $f$ in $G$ has a fixed point (in other words we can find $x_f$ such that $f(x_f)=x_f$. Prove that all the functions in $G$ have a common fixed point.
\vskip 12pt
\noindent {\bf B3}. $a_1,a_2,\ldots,a_n$ are positive reals, and $q$ satisfies $0<q<1$. Find $b_1,b_2,\ldots,b_n$, such that:

(a) $a_i<b_i$ for $i=1,2,\ldots,n$,

(b) $q<{b_{i+1}\over b_i}<{1\over q}$ for $i=1,2,\ldots,n-1$,

(c) $b_1+b_2+\ldots+b_n<(a_1+a_2+\ldots+a_n){1+q\over1-q}$.
\vskip 40pt
\noindent {\bf 16th IMO 1974}
\vskip 25pt
\noindent {\bf A1}. Three players play the following game. There are three cards each with a different positive integer. In each round, the cards are randomly dealt to the players and each receives the number of counters on his card. After two or more rounds, one player has received $20$, another $10$ and the third $9$ counters. In the last round the player with $10$ received the largest number of counters. Who received the middle number on the first round?
\vskip 12pt
\noindent {\bf A2}. Prove that there is a point $D$ on the side $AB$ of the triangle $ABC$, such that $CD$ is the geometric mean of $AD$ and $DB$ iff $\sin A\sin B\le\sin^2{C\over2}$.
\vskip 12pt
\noindent {\bf A3}. Prove that $\sum_0^n{2n+1\choose2k+1}2^{3k}$ is not divisible by $5$ for any non-negative integer $n$.
\vskip 12pt
\noindent {\bf B1}. An $8\times8$ chessboard is divided into $p$ disjoint rectangles (along the lines between the squares), so that each rectangle has the same number of white squares as black squares, and each rectangle has a different number of squares. Find the maximum possible value of $p$ and all possible sets of rectangle sizes.
\vskip 12pt
\noindent {\bf B2}. Determine all possible values of ${a\over a+b+d}+{b\over a+b+c}+{c\over b+c+d}+{d\over a+c+d}$ for positive reals $a,b,c,d$.
\vskip 12pt
\noindent {\bf B3}. Let $P(x)$ be a polynomial with integer coefficients of degree $d>0$. Let $n$ be the number of distinct integer roots to $P(x)=1$ or $-1$. Prove that $n\le d+2$.
\vskip 40pt
\noindent {\bf 17th IMO 1975}
\vskip 25pt
\noindent {\bf A1}. Let $x_1\ge x_2\ge\ldots\ge x_n$, and $y_1\ge y_2\ge\ldots\ge y_n$ be real numbers. Prove that if $z_i$ is any permutation of the $y_i$, then: $$\sum_1^n(x_i-y_i)^2\le\sum_1^n(x_i-z_i)^2.$$
\vskip 12pt
\noindent {\bf A2}. Let $a_1<a_2<a_3<\ldots$ be positive integers. Prove that for every $i\ge1$, there are infinitely many $a_n$ that can be written in the form $a_n=ra_i+sa_j$, with $r,s$ positive integers and $j>i$.
\vskip 12pt
\noindent {\bf A3}. Given any triangle $ABC$, construct external triangles $ABR,BCP,CAQ$ on the sides, so that $\angle PBC=45^o,\angle PCB=30^o,\angle QAC=45^o,\angle QCA=30^o,\angle RAB=15^o,\angle RBA=15^o$. Prove that $\angle QRP=90^o$ and $QR=RP$.
\vskip 12pt
\noindent {\bf B1}. Let $A$ be the sum of the decimal digits of $4444^{4444}$, and $B$ the sum of the decimal digits of $A$. Find the sum of the decimal digits of $B$.
\vskip 12pt
\noindent {\bf B2}. Find $1975$ points on the circumference of a unit circle such that the distance between each pair is rational, or prove it impossible.
\vskip 12pt
\noindent {\bf B3}. Find all polynomials $P(x,y)$ in two variables such that:

(1) $P(tx,ty)=t^nP(x,y)$ for some positive integer $n$ and all real $t,x,y$:

(2) for all real $x,y,z: P(y+z,x)+P(z+x,y)+P(x+y,z)=0$;

(3) $P(1,0)=1$.
\vskip 40pt
\noindent {\bf 18th IMO 1976}
\vskip 25pt
\noindent {\bf A1}. A plane convex quadrilateral has area $32$, and the sum of two opposite sides and a diagonal is $16$. Determine all possible lengths for the other diagonal.
\vskip 12pt
\noindent {\bf A2}. Let $P_1(x)=x^2-2$, and $P_{i+1}=P_1(P_i(x))$ for $i=1,2,3,\ldots$. Show that the roots of $P_n(x)=x$ are real and distinct for all $n$.
\vskip 12pt
\noindent {\bf A3}. A rectangular box can be completely filled with unit cubes. If one places as many cubes as possible, each with volume 2, in the box, with their edges parallel to the edges of the box, one can fill exactly 40\% of the box. Determine the possible dimensions of the box.
\vskip 12pt
\noindent {\bf B1}. Determine the largest number which is the product of positive integers with sum $1976$.
\vskip 12pt
\noindent {\bf B2}. $n$ is a positive integer and $m=2n$. $a_{ij}=0,1$ or $-1$ for $1\le i\le n,1\le j\le m$. The $m$ unknowns $x_1,x_2,\ldots,x_m$ satisfy the $n$ equations: $$a_{i1}x_1+a_{i2}x_2+\ldots+a_{im}x_m=0,$$ for $i=1,2,\ldots,n$. Prove that the system has a solution in integers of absolute value at most $m$, not all zero.
\vskip 12pt
\noindent {\bf B3}. The sequence $u_0,u_1,u_2,\ldots$ is defined by: $u_0,u_1={5\over2},u_{n+1}=u_n(u_{n-1}^2-2)-u_1$ for $n=1,2,\ldots$. Prove that $[u_n]=2^{2^n-(-1)^n\over3}$, where $[x]$ denotes the greatest integer less than or equal to $x$.
\vskip 40pt
\noindent {\bf 19th IMO 1977}
\vskip 25pt
\noindent {\bf A1}. Construct equilateral triangles $ABK, BCL, CDM, DAN$ on the inside of a square $ABCD$. Show that the midpoints of $KL,LM,MN,NK$ and the midpoints of $AK,BK,BL,CL,CM,DM,DN,AN$ form a regular dodecahedron.
\vskip 12pt
\noindent {\bf A2}. In a finite sequence of real numbers the sum of any seven successive terms is negative, and the sum of any eleven successive terms is positive. Determine the maximum number of terms in the sequence.
\vskip 12pt
\noindent {\bf A3}. Given an integer $n>2$, let $V_n$ be the set of integers $1+kn$ for $k$ a positive integer. A number $m$ in $V_n$ is called indecomposable if it cannot be expressed as the product of two members of $V_n$. Prove that there is a number in $V_n$ which can be expressed as the product of indecomposable members of $V_n$ in more than one way (decompositions which differ solely in the order of factors are not regarded as different).
\vskip 12pt
\noindent {\bf B1}. Define $f(x)=1-a\cos x-b\sin x-A\cos2x-B\sin2x$, where $a,b,A,B$ are real constants. Suppose $f(x)\ge)$ for all real $x$. Prove that $a^2+b^2\le2$ and $A^2+B^2\le1$.
\vskip 12pt
\noindent {\bf B2}. Let $a$ and $b$ be positive integers. When $a^2+b^2$ is divided by $a+b$, the quotient is $q$ and the remainder is $r$. Find all pairs $a,b$ such that $q^2+r=1977$.
\vskip 12pt
\noindent {\bf B3}. The function $f$ is defined on the set of positive integers and its values are positive integers. Given that $f(n+1)>f(f(n))$ for all $n$, prove that $f(n)=n$ for all $n$.
\vskip 40pt
\noindent {\bf 20th IMO 1978}
\vskip 25pt
\noindent {\bf A1}. $m$ and $n$ are positive integers with $m<n$. The last three decimal digits of $1978^m$ are the same as the last three decimal digits of $1978^n$. Find $m$ and $n$ such that $m+n$ has the least possible value.
\vskip 12pt
\noindent {\bf A2}. $P$ is a point inside a sphere. Three mutually perpendicular rays from $P$ intersect the sphere at points $U,V$ and $W$. $Q$ denotes the vertex diagonally oppposite $P$ in the parallelepiped determined by $PU,PV,PW$. Find the locus of $Q$ for all possible sets of such rays from $P$.
\vskip 12pt
\noindent {\bf A3}. The set of all positive integers is the union of two disjoint subsets $\{f(1),f(2),f(3),\ldots\}$, $\{g(1),g(2),g(3),\ldots\}$, where $f(1)<f(2)<\ldots$, and $g(1)<g(2)<g(3)<\ldots$, and $g(n)=f(f(n))+1$ for $n=1,2,3,\ldots$. Determine $f(240)$.
\vskip 12pt
\noindent {\bf B1}. In the triangle $ABC, AB=AC$. A circle is tangent internally to the circumcircle of the triangle and also to $AB,AC$ at $P,Q$ respectively. Prove that the midpoint of $PQ$ is the center of the incircle of the triangle.
\vskip 12pt
\noindent {\bf B2}. $\{a_k\}$ is a sequence of distinct positive integers. Prove that for all positive integers $n, \sum_1^n{a_k\over k^2}\ge\sum_1^n{1\over k}$.
\vskip 12pt
\noindent {\bf B3}. An international society has its members from six different countries. The list of members has $1978$ names, numbered $1,2,\ldots,1978$. Prove that there is at least one member whose number is the sum of the numbers of two members from his own country, or twice the number of a member from his own country.
\vskip 40pt
\noindent {\bf 21st IMO 1979}
\vskip 25pt
\noindent {\bf A1}. Let $m$ and $n$ be positive integers such that $${m\over n}=1-{1\over2}+{1\over3}-{1\over4}+\ldots-{1\over1318}+{1\over1319}.$$ Prove that $m$ is divisible by $1979$.
\vskip 12pt
\noindent {\bf A2}. A prism with pentagons $A_1A_2A_3A_4A_5$ and $B_1B_2B_3B_4B_5$ as the top and bottom faces is given. Each side of the two pentagons and each of the $25$ segments $A_iB_j$ is colored red or green. Every triangle whose vertices are vertices of the prism and whose sides have all been colored has two sides of a different color. Prove that all $10$ sides of the top and bottom faces have the same color.
\vskip 12pt
\noindent {\bf A3}. Two circles in a plane intersect. $A$ is one of the points of intersection. Starting simultaneously from $A$ two points move with constant speed, each traveling along is own circle in the same sense. The two points return to $A$ simultaneously after one revolution. Prove that there is a fixed point $P$ in the plane such that the two points are always equidistant from $P$.
\vskip 12pt
\noindent {\bf B1}. Given a plane $k$, a point $P$ in the plane and a point $Q$ not in the plane, find all points $R$ in $k$ such that the ratio $QP+PR\over QR$ is a maximum.
\vskip 12pt
\noindent {\bf B2}. Find all real numbers $a$ for which there exist non-negative real numbes $x_1,x_2,x_3,x_4,x_5$ satisfying:

$x_1+2x_2+3x_3+4x_4+5x_5=a,$

$x_1+2^3x_2+3^3x_3+4^3x_4+5^3x_5=a^2,$

$x_1+2^5x_2+3^5x_3+4^5x_4+5^5x_5=a^3.$
\vskip 12pt
\noindent {\bf B3}. Let $A$ and $E$ be opposite vertices of an octagon. A frog starts at vertex $A$. From any vertex except $E$ it jumps to one of the two adjacent vertices. When it reaches $E$ it stops. Let $a_n$ be the number of distinct paths of exactly $n$ jumps ending at $E$. Prove that: $a_{2n-1}=0,a_{2n}={(2+\sqrt2)^{n-1}\over\sqrt2}-{(2-\sqrt2)^{n-1}\over\sqrt2}$.
\vskip 40pt
\noindent {\bf 22nd IMO 1981}
\vskip 25pt
\noindent {\bf A1}. $P$ is a point inside the triangle $ABC$. $D,E,F$ are the feet of the perpendiculars from $P$ to the lines $BC,CA,AB$ respectively. Find all $P$ which minimize: $${BC\over PD}+{CA\over PE}+{AB\over PF}.$$
\vskip 12pt
\noindent {\bf A2}. Take $r$ such that $1\le r\le n$, and consider all subsets of $r$ elements of the set $\{1,2,\ldots,n\}$. Each subset has a smallest element. Let $F(n,r)$ be the arithmetic mean of these smallest elements. Prove that: $$F(n,r)={n+1\over r+1}.$$
\vskip 12pt
\noindent {\bf A3}. Determine the maximum value of $m^2+n^2$, where $m$ and $n$ are integers in the range $1,2,\ldots,1981$ satisfying $(n^2-mn-m^2)^2=1$.
\vskip 12pt
\noindent {\bf B1}. (a) For which $n>2$ is there a set of $n$ consecutive positive integers such that the largest number in the set is a divisor of the least common multiple of the remaining $n-1$ numbers?

(b) For which $n>2$ is there exactly one set having this property?
\vskip 12pt
\noindent {\bf B2}. Three circles of equal radius have a common point $O$ and lie inside a given triangle. Each circle touches a pair of sides of the triangle. Prove that the incenter and the circumcenter of the triangle are collinear with the point $O$.
\vskip 12pt
\noindent {\bf B3}. The function $f(x,y)$ satisfies: $f(0,y)=y+1, f(x+1,0) = f(x,1), f(x+1,y+1)=f(x,f(x+1,y))$ for all non-negative integers $x,y$. Find $f(4,1981)$.
\vskip 40pt
\noindent {\bf 23rd IMO 1982}
\vskip 25pt
\noindent {\bf A1}. The function $f(n)$ is defined on the positive integers and takes non-negative integer values. $f(2)=0,f(3)>0,f(9999)=3333$ and for all $m,n$: $f(m+n)-f(m)-f(n)=0$ or $1$. Determine $f(1982)$.
\vskip 12pt
\noindent {\bf A2}. A non-isosceles triangle $A_1A_2A_3$ has sides $a_1,a_2,a_3$ with $a_i$ opposite $A_i$. $M_i$ is the midpoint of side $a_i$ and $T_i$ is the point where the incircle touches side $a_i$. Denote by $S_i$ the reflection of $T_i$ in the interior bisector of angle $A_i$. Prove that the lines $M_1S_1,M_2S_2$ and $M_3S_3$ are concurrent.
\vskip 12pt
\noindent {\bf A3}. Consider infinite sequences $\{x_n\}$ of positive reals such that $x_0=1$ and $x_0\ge x_1\ge x_2\ge\ldots$.

(a) Prove that for every such sequence there is an $n\ge1$ such that: $${x_0^2\over x_1}+{x_1^2\over x_2}+\ldots+{x_{n-1}^2\over x_n}\ge3.999.$$

(b) Find such a sequence such that for all $n$: $${x_0^2\over x_1}+{x_1^2\over x_2}+\ldots+{x_{n-1}^2\over x_n}<4.$$
\vskip 12pt
\noindent {\bf B1}. Prove that if $n$ is a positive integer such that the equation $$x^3-3xy^2+y^3=n$$
has a solution in integers $x,y$, then it has at least three such solutions. Show that the equation has no solutions in integers for $n=2891$.
\vskip 12pt
\noindent {\bf B2}. The diagonals $AC$ and $CE$ of the regular hexagon $ABCDEF$ are divided by inner points $M$ and $N$ respectively, so that $${AM\over AC}={CN\over CE}=r.$$ Determine $r$ if $B,M$ and $N$ are collinear.
\vskip 12pt
\noindent {\bf B3}. Let $S$ be a square with sides length $100$. Let $L$ be a path within $S$ which does not meet itself and which is composed of line segments $A_0A_1,A_1A_2,A_2A_3,\ldots,A_{n-1}A_n$ with $A_0=A_n$. Suppose that for every point $P$ on the boundary of $S$ there is a point of $L$ at a distance from $P$ no greater than $1\over2$. Prove that there are two points $X$ and $Y$ of $L$ such that the distance between $X$ and $Y$ is not greater than $1$ and the length of the part of $L$ which lies between $X$ and $Y$ is not smaller than $198$.
\vskip 40pt
\noindent {\bf 24th IMO 1983}
\vskip 25pt
\noindent {\bf A1}. Find all functions $f$ defined on the set of positive reals which take positive real values and satisfy: 

$f(xf(y))=yf(x)$ for all $x,y$; and $f(x)\to0$ as $x\to\infty$.
\vskip 12pt
\noindent {\bf A2}. Let $A$ be one of the two distinct points of intersection of two unequal coplanar circles $C_1$ and $C_2$ with centers $O_1$ and $O_2$ respectively. One of the common tangents to the circles touches $C_1$ at $P_1$ and $C_2$ at $P_2$, while the other touches $C_1$ at $Q_1$ and $C_2$ at $Q_2$. Let $M_1$ be the midpoint of $P_1Q_1$ and $M_2$ the midpoint of $P_2Q_2$. Prove that $\angle O_1AO_2=\angle M_1AM_2$.
\vskip 12pt
\noindent {\bf A3}. Let $a,b$ and $c$ be positive integers, no two of which have a common divisor greater than $1$. Show that $2abc-ab-bc-ca$ is the largest integer which cannot be expressed in the form $xbc+yca+zab$, where $x,y,z$ are non-negative integers.
\vskip 12pt
\noindent {\bf B1}. Let $ABC$ be an equilateral triangle and $E$ the set of all points contained in the three segments $AB,BC$ and $CA$ (including $A,B$ and $C$). Determine whether, for every partition of $E$ into two disjoint subsets, at least one of the two subsets contains the vertices of a right-angled triangle.
\vskip 12pt
\noindent {\bf B2}. Is it possible to choose $1983$ distinct positive integers, all less than or equal to $10^5$, no three of which are consecutive terms of an arithmetic progression?
\vskip 12pt
\noindent {\bf B3}. Let $a,b$ and $c$ be the lengths of the sides of a triangle. Prove that $$a^2b(a-b)+b^2c(b-c)+c^2a(c-a)\ge0.$$ Determine when equality occurs.
\vskip 40pt
\noindent {\bf 25th IMO 1984}
\vskip 25pt
\noindent {\bf A1}. Prove that $0\le yz+zx+xy-2xyz\le{7\over27}$, where $x,y$ and $z$ are non-negative real numbers satisfying $x+y+z=1$.
\vskip 12pt
\noindent {\bf A2}. Find one pair of positive integers $a,b$ such that $ab(a+b)$ is not divisible by $7$, but $(a+b)^7-a^7-b^7$ is divisible by $7^7$.
\vskip 12pt
\noindent {\bf A3}. Given points $O$ and $A$ in the plane. Every point in the plane is colored with one of a finite number of colors. Given a point $X$ in the plane, the circle $C(X)$ has center $O$ and radius $OX+{\angle AOX\over OX}$, where $\angle AOX$ is measured in radians in the range $[0,2\pi)$. Prove that we can find a point $X$, not on $OA$, such that its color appears on the circumference of the circle $C(X)$.
\vskip 12pt
\noindent {\bf B1}. Let $ABCD$ be a convex quadrilateral with the line $CD$ tangent to the circle on diameter $AB$. Prove that the line $AB$ is tangent to the circle on diameter $CD$ iff $BC$ and $AD$ are parallel.
\vskip 12pt
\noindent {\bf B2}. Let $d$ be the sum of the lengths of all the diagonals of a plane convex polygon with $n>3$ vertices. Let $p$ be its perimeter. Prove that: $$n-3<{2d\over p}<\Bigl[{n\over2}\Bigr]\Bigl[{n+1\over 2}\Bigr]-2,$$ where $[x]$ denotes the greatest integer not exceeding $x$.
\vskip 12pt
\noindent {\bf B3}. Let $a,b,c,d$ be odd integers such that $0<a<b<c<d$ and $ad=bc$. Prove that if $a+d=2^k$ and $b+c=2^m$ for some integers $k$ and $m$, then $a=1$.
\vskip 40pt
\noindent {\bf 26th IMO 1985}
\vskip 25pt
\noindent {\bf A1}. A circle has center on the side $AB$ of the cyclic quadrilateral $ABCD$. The other three sides are tangent to the circle. Prove that $AD+BC=AB$.
\vskip 12pt
\noindent {\bf A2}. Let $n$ and $k$ be relatively prime positive integers with $k<n$. Each number in the set $M=\{1,2,3,\ldots,n-1\}$ is colored either blue or white. For each $i$ in $M$, both $i$ and $n-i$ have the same color. For each $i\ne k$ in $M$ both $i$ and $|i-k|$ have the same color. Prove that all numbers in $M$ must have the same color.
\vskip 12pt
\noindent {\bf A3}. For any polynomial $P(x)=a_0+a_1x+\ldots+a_kx^k$ with integer coefficients, the number of odd coefficients is denoted by $o(P)$. For $i-0,1,2,\ldots$ let $Q_i(x)=(1+x)^i$. Prove that if $i_1,i_2,\ldots,i_n$ are integers satisfying $0\le i_1<i_2<\ldots<i_n$, then: $$o(Q_{i_1}+Q_{i_2}+\ldots+Q_{i_n})\ge o(Q_{i_1}).$$
\vskip 12pt
\noindent {\bf B1}. Given a set $M$ of $1985$ distinct positive integers, none of which has a prime divisor greater than $23$, prove that $M$ contains a subset of $4$ elements whose product is the $4$th power of an integer.
\vskip 12pt
\noindent {\bf B2}. A circle center $O$ passes through the vertices $A$ and $C$ of the triangle $ABC$ and intersects the segments $AB$ and $BC$ again at distinct points $K$ and $N$ respectively. The circumcircles of $ABC$ and $KBN$ intersect at exactly two distinct points $B$ and $M$. Prove that $\angle OMB$ is a right angle.
\vskip 12pt
\noindent {\bf B3}. For every real number $x_1$, construct the sequence $x_1,x_2,\ldots$ by setting: $$x_{n+1}=x_n(x_n+{1\over n}).$$ Prove that there exists exactly one value of $x_1$ which gives $0<x_n<x_{n+1}<1$ for all $n$.
\vskip 40pt
\noindent {\bf 27th IMO 1986}
\vskip 25pt
\noindent {\bf A1}. Let $d$ be any positive integer not equal to $2, 5$ or $13$. Show that one can find distinct $a,b$ in the set $\{2,5,13,d\}$ such that $ab-1$ is not a perfect square.
\vskip 12pt
\noindent {\bf A2}. Given a point $P_0$ in the plane of the triangle $A_1A_2A_3$. Define  $A_s=A_{s-3}$ for all $s\ge4$. Construct a set of points $P_1,P_2,P_3,\ldots$ such that $P_{k+1}$ is the image of $P_k$ under a rotation center $A_{k+1}$ through an angle $120^o$ clockwise for $k=0,1,2,\ldots$. Prove that if $P_{1986}=P_0$, then the triangle $A_1A_2A_3$ is equilateral.
\vskip 12pt
\noindent {\bf A3}. To each vertex of a regular pentagon an integer is assigned, so that the sum of all five numbers is positive. If three consecutive vertices are assigned the numbers $x,y,z$ respectively, and $y<0$, then the following operation is allowed: $x,y,z$ are replaced by $x+y,-y,z+y$ respectively. Such an operation is performed repeatedly as long as at least one of the five numbers is negative. Determine whether this procedure necessarily comes to an end after a finite number of steps.
\vskip 12pt
\noindent {\bf B1}. Let $A,B$ be adjacent vertices of a regular $n$-gon ($n\ge5$) with center $O$. A triangle $XYZ$, which is congruent to and initially coincides with $OAB$, moves in the plane in such a way that $Y$ and $Z$ each trace out the whole boundary of the polygon, with $X$ remaining inside the polygon. Find the locus of $X$.
\vskip 12pt
\noindent {\bf B2}. Find all functions $f$ defined on the non-negative reals and taking non-negative real values such that: $f(2)=0,f(x)\ne0$ for $0\le x<2$, and $f(xf(y))f(y)=f(x+y)$ for all $x,y$.
\vskip 12pt
\noindent {\bf B3}. Given a finite set of points in the plane, each with integer coordinates, is it always possible to color the points red or white so that for any straight line $L$ parallel to one of the coordinate axes the difference (in absolute value) between the numbers of white and red points on $L$ is not greater than $1$?
\vskip 40pt
\noindent {\bf 28th IMO 1987}
\vskip 25pt
\noindent {\bf A1}. Let $p_n(k)$ be the number of permutations of the set $\{1,2,3,\ldots,n\}$ which have exactly $k$ fixed points. Prove that $\sum_0^nk p_n(k)=n!$.
\vskip 12pt
\noindent {\bf A2}. In an acute-angled triangle $ABC$ the interior bisector of angle $A$ meets $BC$ at $L$ and meets the circumcircle of $ABC$ again at $N$. From $L$ perpendiculars are drawn to $AB$ and $AC$, with feet $K$ and $M$ respectively. Prove that the quadrilateral $AKNM$ and the triangle $ABC$ have equal areas.
\vskip 12pt
\noindent {\bf A3}. Let $x_1,x_2,\ldots,x_n$ be real numbers satisfying $x_1^2+x_2^2+\ldots+x_n^2=1$. Prove that for every integer $k\ge2$ there are integers $a_1,a_2,\ldots,a_n$, not all zero, such that $|a_i|\le k-1$ for all $i$, and $|a_1x_1+a_2x_2+\ldots+a_nx_n|\le{(k-1)\sqrt n\over k^n-1}$.
\vskip 12pt
\noindent {\bf B1}. Prove that there is no function $f$ from the set of non-negative integers into itself such that $f(f(n))=n+1987$ for all $n$.
\vskip 12pt
\noindent {\bf B2}. Let $n\ge3$ be an integer. Prove that there is a set of $n$ points in the plane such that the distance between any two points is irrational and each set of three points determines a non-degenerate triangle with rational area. 
\vskip 12pt
\noindent {\bf B3}. Let $n\ge2$ be an integer. Prove that if $k^2+k+n$ is prime for all integers $k$ such that $0\le k\le\sqrt{n\over3}$, then $k^2+k+n$ is prime for all integers $k$ such that $0\le k\le n-2$.
\vskip 40pt
\noindent {\bf 29th IMO 1988}
\vskip 25pt
\noindent {\bf A1}. Consider two coplanar circles of radii $R>r$ with the same center. Let $P$ be a fixed point on the smaller circle and $B$ a variable point on the larger circle. The line $BP$ meets the larger circle again at $C$. The perpendicular to $BP$ at $P$ meets the smaller circle again at $A$ (if it is tangent to the circle at $P$, then $A=P$).

(i) Find the set of values of $AB^2+BC^2+CA^2$.

(ii) Find the locus of the midpoint of $BC$.
\vskip 12pt
\noindent {\bf A2}. Let $n$ be a positive integer and let $A_1,A_2,\ldots,A_{2n+1}$ be subsets of a set $B$. Suppose that:

(i) Each $A_i$ has exactly $2n$ elements,

(ii) The intersection of every two distinct $A_i$ contains exactly one element, and

(iii) Every element of $B$ belongs to at least two of the $A_i$.

\noindent For which values of $n$ can one assign to every element of $B$ one of the numbers $0$ and $1$ in such a way that each $A_i$ has $0$ assigned to exactly $n$ of its elements?
\vskip 12pt
\noindent {\bf A3}. A function $f$ is defined on the positive integers by: $f(1)=1, f(3)=3, f(2n)=f(n),f(4n+1)=2f(2n+1)-f(n)$, and $f(4n+3)=3f(2n+1)-2f(n)$ for all positive integers $n$. Determine the number of positive integers $n\le1988$ for which $f(n)=n$.
\vskip 12pt
\noindent {\bf B1}. Show that the set of real numbers $x$ which satisfy the inequality: $${1\over x-1}+{2\over x-2}+{3\over x-3}+\ldots+{70\over x-70}\ge{5\over4}$$ is a union of disjoint intervals, the sum of whose lengths is $1988$.
\vskip 12pt
\noindent {\bf B2}. $ABC$ is a triangle, right-angled at $A$, and $D$ is the foot of the altitude from $A$. The straight line joining the incenters of the triangles $ABD$ and $ACD$ intersects the sides $AB,AC$ at $K,L$ respectively. Show that the area of the triangle $ABC$ is at least twice the area of the triangle $AKL$.
\vskip 12pt
\noindent {\bf B3}. Let $a$ and $b$ be positive integers such that $ab+1$ divides $a^2+b^2$. Show that $a^2+b^2\over ab+1$ is a perfect square.
\vskip 40pt
\noindent {\bf 30th IMO 1989}
\vskip 25pt
\noindent {\bf A1}. Prove that the set $\{1,2,\ldots,1989\}$ can be expressed as the disjoint union of subsets $A_1,A_2,\ldots,A_{117}$ in such a way that each $A_i$ contains $17$ elements and the sum of the elements in each $A_i$ is the same.
\vskip 12pt
\noindent {\bf A2}. In an acute-angled triangle $ABC$, the internal bisector of angle $A$ meets the circumcircle again at $A_1$. Points $B_1$ and $C_1$ are defined similarly. Let $A_0$ be the point of intersection of the line $AA_1$ with the external bisectors of angles $B$ and $C$. Points $B_0$ and $C_0$ are defined similarly. Prove that the area of the triangle $A_0B_0C_0$ is twice the area of the hexagon $AC_1BA_1CB_1$ and at least four times the area of the triangle $ABC$.
\vskip 12pt
\noindent {\bf A3}. Let $n$ and $k$ be positive integers, and let $S$ be a set of $n$ points in the plane such that no three points of $S$ are collinear, and for any points $P$ of $S$ there are at least $k$ points of $S$ equidistant from $P$. Prove that $k<{1\over2}+\sqrt{2n}$.
\vskip 12pt
\noindent {\bf B1}. Let $ABCD$ be a convex quadrilateral such that the sides $AB,AD,BC$ satisfy $AB=AD+BC$. There exists a point $P$ inside the quadrilateral at a distance $h$ from the line $CD$ such that $AP=h+AD$ and $BP=h+BC$. Show that: $${1\over\sqrt h}\ge{1\over\sqrt{AD}}+{1\over\sqrt{BC}}.$$
\vskip 12pt
\noindent {\bf B2}. Prove that for each positive integer $n$ there exist $n$ consecutive positive integers none of which is a prime or a prime power.
\vskip 12pt
\noindent {\bf B3}. A permutation $\{x_1,x_2,\ldots,x_m\}$ of the set $\{1,2,\ldots,2n\}$ where $n$ is a positive integer is said to have property $P$ if $|x_i-x_{i+1}|=n$ for at least one $i$ in $\{1,2,\ldots,2n-1\}$. Show that for each $n$ there are more permutations with property $P$ than without.
\vskip 40pt
\noindent {\bf 31st IMO 1990}
\vskip 25pt
\noindent {\bf A1}. Chords $AB$ and $CD$ of a circle intersect at a point $E$ inside the circle. Let $M$ be an interior point of the segment $EB$. The tangent at $E$ to the circle through $D,E,M$ intersects the lines $BC$ and $AC$ at $F$ and $G$ respectively. Find $EF\over EF$ in terms of $t={AM\over AB}$.
\vskip 12pt
\noindent {\bf A2}. Take $n\ge3$ and consider a set $E$ of $2n-1$ distinct points on a circle. Suppose that exactly $k$ of these points are to be colored black. Such a coloring is {\it good} if there is at least one pair of black points such that the interior of one of the arcs between them contains exactly $n$ points from $E$. Find the smallest value of $k$ so that every such coloring of $k$ points of $E$ is good.
\vskip 12pt
\noindent {\bf A3}. Determine all integers greater than $1$ such that $2^n+1\over n^2$ is an integer.
\vskip 12pt
\noindent {\bf B1}. Construct a function from the set of positive rational numbers into itself such that $f(xf(y))={f(x)\over y}$ for all $x,y$.
\vskip 12pt
\noindent {\bf B2}. Given an initial integer $n_0>1$, two players $A$ and $B$ choose integers $n_1,n_2,n_3,\ldots$ alternately according to the following rules:

Knowing $n_{2k}, A$ chooses any integer $n_{2k+1}$ such that $n_{2k}\le n_{2k+1}\le n_{2k}^2$.

Knowing $n_{2k+1},B$ chooses any integer $n_{2k+2}$ such that ${n_{2k+1}\over n_{2k+2}}=p^r$ for some prime $p$ and integer $r\ge 1$.

\noindent Player $A$ wins the game by choosing the number $1990$; player $B$ wins by choosing the number $1$. For which $n_0$ does

(a) $A$ have a winning strategy?

(b) $B$ have a winning strategy?

(c) Neither player have a winning strategy?
\vskip 12pt
\noindent {\bf B3}. Prove that there exists a convex $1990$-gon such that all its angles are equal and the lengths of the sides are the numbers $1^2,2^2,\ldots, 1990^2$ in some order.
\vskip 40pt
\noindent {\bf 32nd IMO 1991}
\vskip 25pt
\noindent {\bf A1}. Given a triangle $ABC$, let $I$ be the incenter. The internal bisectors of angles $A,B,C$ meet the opposite sides in $A',B',C'$ respectively. Prove that $${1\over4}<{AI\cdot BI\cdot CI\over AA'\cdot BB'\cdot CC'}\le {8\over27}.$$
\vskip 12pt
\noindent {\bf A2}. Let $n>6$ be an integer and let $a_1,a_2,\ldots ,a_k$ be all the positive integers less than $n$ and relatively prime to $n$. If $$a_2-a_1=a_3-a_2=\ldots a_k-a_{k-1}>0,$$ prove that $n$ must be either a prime or a power of $2$.
\vskip 12pt
\noindent {\bf A3}. Let $S=\{1,2,3,\ldots ,280\}$. Find the smallest integer $n$ such that each $n$-element subset of $S$ contains five numbers which are pairwise relatively prime.
\vskip 12pt
\noindent {\bf B1}. Suppose $G$ is a connected graph with $k$ edges. Prove that it is possible to label the edges $1,2,\ldots ,k$ in such a way that at each vertex which belongs to two or more edges, the greatest common divisor of the integers labeling those edges is $1$.
\vskip 12pt
\noindent {\bf B2}. Let $ABC$ be a triangle and $X$ an interior point of $ABC$. Show that at least one of the angles $XAB,XBC,XCA$ is less than or equal to $30^o$.
\vskip 12pt
\noindent {\bf B3}. Given any real number $a>1$ construct a bounded infinite sequence $x_0,x_1,x_2,\ldots$ such that $|x_i-x_j||i-j|^a\ge 1$ for every pair of distinct $i,j$.
\vskip 40pt
\noindent {\bf 33rd IMO 1992}
\vskip 25pt
\noindent {\bf A1}. Find all integers $a,b,c$ satisfying $1<a<b<c$ such that $(a-1)(b-1)(c-1)$ is a divisor of $abc-1$.
\vskip 12pt
\noindent {\bf A2}. Find all functions $f$ defined on the set of all real numbers with real values, such that $f(x^2+f(y))=y+f(x)^2$ for all $x,y$.
\vskip 12pt
\noindent {\bf A3}. Consider $9$ points in space, no $4$ coplanar. Each pair of points is joined by a line segment which is colored either blue or red or left uncolored. Find the smallest value of $n$ such that whenever exactly $n$ edges are colored, the set of colored edges necessarily contains a triangle all of whose edges have the same color.
\vskip 12pt
\noindent {\bf B1}. $L$ is a tangent to the circle $C$ and $M$ is a point on $L$. Find the locus of all points $P$ such that there exist points $Q$ and $R$ on $L$ equidistant from $M$ with $C$ the incircle of the triangle $PQR$.
\vskip 12pt
\noindent {\bf B2}. Let $S$ be a finite set of points in three-dimensional space. Let $S_x,S_y,S_z$ be the sets consisting of the orthogonal projections of the points of $S$ onto the $yz$-plane, $zx$-plane, $xy$-plane respectively. Prove that: $$|S|^2\le |S_x||S_y||S_z|,$$ where $|A|$ denotes the number of points in the set $A$. The orthogonal projection of a point onto a plane is the foot of the perpendicular from the point to the plane.
\vskip 12pt
\noindent {\bf B3}. For each positive integer $n,S(n)$ is defined as the greatest integer such that for every positive integer $k\le S(n),n^2$ can be written as the sum of $k$ positive squares.

(a) Prove that $S(n)\le n^2-14$ for each $n\ge4$.

(b) Find an integer $n$ such that $S(n)=n^2-14$.

(c) Prove that there are infinitely many integers $n$ such that $S(n)=n^2-14$.
\vskip 40pt
\noindent {\bf 34th IMO 1993}
\vskip 25pt
\noindent {\bf A1}. Let $f(x)=x^n+5x^{n-1}+3$, where $n>1$ is an integer. Prove that $f(x)$ cannot be expressed as the produce of two non-constant polynomials with integer coefficients.
\vskip 12pt
\noindent {\bf A2}. Let $D$ be a point inside the acute-angled triangle $ABC$ such that $\angle ADB=\angle ACB+90^o$, and $AC\cdot BD=AD\cdot BC$.

(a) Calculate the ratio $AB\cdot CD/(AC\cdot BD)$.

(b) Prove that the tangents at $C$ to the circumcircles of $ACD$ and $BCD$ are perpendicular.
\vskip 12pt
\noindent {\bf A3}. On an infinite chessboard a game is played as follows. At the start $n^2$ pieces are arranged in an $n\times n$ block of adjoining squares, one piece on each square. A move in the game is a jump in a horizontal or vertical direction over an adjacent occupied square to an unoccupied square immediately beyond. The piece which has been jumped over is removed. Find those values of $n$ for which the game can end with only one piece remaining on the board.
\vskip 12pt
\noindent {\bf B1}. For the points $P,Q,R$ in the plane define $m(PQR)$ as the minimum length of the three altitudes of the triangle $PQR$ (or zero if the points are collinear). Prove that for any points $A,B,C,X$: $$m(ABC)\le m(ABX)+m(AXC)+m(XBC).$$
\vskip 12pt
\noindent {\bf B2}. Does there exist a function $f$ from the positive integers to the positive integers such that $f(1)=2,f(f(n))=f(n)+n$ for all $n$, and $f(n)<f(n+1)$ for all $n$?
\vskip 12pt
\noindent {\bf B3}. There are $n>1$ lamps $L_0,L_1,\ldots ,L_{n-1}$ in a circle. We use $L_{n+k}$ to mean $L_k$. A lamp is at all times either on or off. Initially they are all on. Perform steps $s_0,s_1,\ldots$ as follows: at step $s_i$, if $L_{i-1}$ is lit, then switch $L_i$ from on to off or vice versa, otherwise do nothing. Show that:

(a) There is a positive integer $M(n)$ such that after $M(n)$ steps all the lamps are on again;

(b) If $n=2^k$, then we can take $M(n)=n^2-1$;

(c) If $n=2^k+1$, then we can take $M(n)=n^2-n+1$.
\vskip 40pt
\noindent {\bf 35th IMO 1994}
\vskip 25pt
\noindent {\bf A1}. Let $m$ and $n$ be positive integers. Let $a_1, a_2, \ldots , a_m$ be distinct elements of $\{1, 2, \ldots , n\}$ such that whenever $a_i+a_j \le n$ for some $i,j$ (possibly the same) we have $a_i+a_j=a_k$ for some $k$. Prove that: $$(a_1+\ldots a_m)\ge {(n+1) \over 2}.$$
\vskip 12pt
\noindent {\bf A2}. $ABC$ is an isosceles triangle with $AB=AC$, $M$ is the midpoint of $BC$ and $O$ is the point on the line $AM$ such that $OB$ is perpendicular to $AB$. $Q$ is an arbitrary point on $BC$ different from $B$ and $C$. $E$ lies on the line $AB$ and $F$ lies on the line $AC$ such that $E, Q, F$ are distinct and collinear Prove that $OQ$ is perpendicular to $EF$ iff $QE=QF$.
\vskip 12pt
\noindent {\bf A3}.  For any positive integer $k$, let $f(k)$ be the number of elements in the set $\{k+1, k+2, \ldots, 2k\}$ which have exactly three $1$s when written in base $2$. Prove that for each positive integer $m$, there is at least one $k$ with $f(k)=m$, and determine all $m$ for which there is exactly one $k$.
\vskip 12pt
\noindent {\bf B1}. Determine all ordered pairs $(m, n)$ of positive integers for which ${n^3+1 \over mn-1}$ is an integer. 
\vskip 12pt
\noindent {\bf B2}. Let $S$ be the set of all real numbers greater than $-1$. Find all functions $f\colon S\to S$ such that $f\bigl(x+f(y)+xf(y)\bigr) = y+f(x) +yf(x)$ for all x, y, and ${f(x)\over x}$ is strictly increasing on each of the intervals $-1<x<0$ and $0<x$.
\vskip 12pt
\noindent {\bf B3}. Show that there exists a set $A$ of positive integers with the following property: for any infinite set $S$ of primes, there exist two positive integers $m\in A$ and $n\notin A$, each of which is a product of $k$ distinct elements of $S$ for some $k\ge 2$.
\vskip 40pt
\noindent {\bf 36th IMO 1995}
\vskip 25pt
\noindent {\bf A1}. Let $A, B, C, D$ be four distinct points on a line, in that order. The circles with diameter $AC$ and $BD$ intersect at $X$ and $Y$. The line $XY$ meets $BC$ at $Z$. Let $P$ be a point on the line $XY$ other than $Z$. The line $CP$ intersects the circle with diameter $AC$ at $C$ and $M$, and the line $BP$ intersects the circle with diameter $BD$ at $B$ and $N$. Prove that the lines $AM, DN, XY$ are concurrent.
\vskip 12pt
\noindent {\bf A2}. Let $a, b, c$ be positive real numbers with $abc = 1$. Prove that: $${1 \over a^3(b+c)} + {1 \over b^3(c+a)} + {1 \over c^3(a+b)} \ge {3 \over 2}.$$
\vskip 12pt
\noindent {\bf A3}. Determine all integers $n>3$ for which there exist $n$ points $A_1, \ldots , A_n$ in the plane, no three collinear, and real numbers $r_1, \ldots , r_n$ such that for any distinct $i, j, k$, the area of the triangle $A_iA_jA_k$ is $r_i+r_j+r_k$. 
\vskip 12pt
\noindent {\bf B1}. Find the maximum value of $x_0$ for which there exists a sequence $x_0, x_1, \ldots , x_{1995}$ of positive reals with $x_0 = x_{1995}$ such that for $i = 1, \ldots , 1995$: $$x_{i-1}+{2 \over x_{i-1}} = 2x_i+{1 \over x_i}.$$
\vskip 12pt
\noindent {\bf B2}. Let $ABCDEF$ be a convex hexagon with $AB=BC=CD$ and $DE=EF=FA$, such that $\angle BCD = \angle EFA = 60^o$. Suppose that $G$ and $H$ are points in the interior of the hexagon such that $\angle AGB = \angle DHE = 120^o$. Prove that $AG+GB+GH+DH+HE \ge CF$.
\vskip 12pt
\noindent {\bf B3}. Let $p$ be an odd prime number. How many $p$-element subsets $A$ of $\{1, 2, \ldots , 2p\}$ are there, the sum of whose elements is divisible by $p$?
\vskip 40pt
\noindent {\bf 37th IMO 1996}
\vskip 25pt
\noindent {\bf A1}. We are given a positive integer $r$ and a rectangular board divided into $20 \times 12$ unit squares. The following moves are permitted on the board: one can move from one square to another only if the distance between the centers of the two squares is $\sqrt{r}$. The task is to find a sequence of moves leading between two adjacent corners of the board which lie along the long side.

\noindent (a) Show that the task cannot be done if $r$ is divisible by $2�$ or $3$.

\noindent (b) Prove that the task is possible for $r = 73$.

\noindent (c) Can the task be done for $r = 97$?
\vskip 12pt
\noindent {\bf A2}. Let $P$ be a point inside the triangle $ABC$ such that $\angle APB - \angle ACB = \angle APC - \angle ABC$. Let $D, E$ be the incenters of triangles $APB, APC$ respectively. Show that $AP, BD, CE$ meet at a point.
\vskip 12pt
\noindent {\bf A3}. Let $S$ be the set of non-negative integers. Find all functions $f:S\to S$ such that $f(m+f(n)) = f(f(m))+f(n)$ for all $m,n$.
\vskip 12pt
\noindent {\bf B1}. The positive integers $a,b$ are such that $15a+16b$ and $16a-15b$ are both squares of positive integers. What is the least possible value that can be taken by the smaller of these two squares?
\vskip 12pt
\noindent {\bf B2}. Let $ABCDEF$ be a convex hexagon such that $AB$ is parallel to $DE, BC$ is parallel to $EF$, and $CD$ is parallel to $FA$. Let $R_A, R_C, R_E$ denote the circumradii of triangles $FAB, BCD, DEF$ respectively, and let $p$ denote the perimeter of the hexagon. Prove that: $$R_A+R_C+R_E\ge p/2$$.
\vskip 12pt
\noindent {\bf B3}. Let $p,q,n$ be three positive integers with $p+q < n$. Let $x_0, x_1, \ldots , x_n$ be integers such that $x_0 = x_n = 0$, and for each $1\le i \le n, x_i - x_{i-1} = p$ or $-q$. Show that there exist indices $i<j$ with $(i,j)$ not $(0,n)$ such that $x_i=x_j$.
\vskip 40pt
\noindent {\bf 38th IMO 1997}
\vskip 25pt
\noindent {\bf A1}. In the plane the points with integer coordinates are the vertices of unit squares. The squares are colored alternately black and white as on a chessboard. For any pair of positive integers $m$ and $n$, consider a right-angled triangle whose vertices have integer coordinates and whose legs, of lengths $m$ and $n$, lie along the edges of the squares. Let $S_1$ be the total area of the black part of the triangle, and $S_2$ be the total area of the white part. Let $f(m, n) = |S_1 - S_2|$.

(a) Calculate $f(m,n)$ for all positive integers $m$ and $n$ which are either both even or both odd.

(b) Prove that $f(m,n) \le$ max$(m,n)/2$ for all $m$, $n$.

(c) Show that there is no constant $C$ such that $f(m,n) < C$ for all $m$, $n$.
\vskip 12pt
\noindent {\bf A2}. $\angle A$ is the smallest angle in the triangle $ABC$. The points $B$ and $C$ divide the circumcircle of the triangle into two arcs. Let $U$ be an interior point of the arc between $B$ and $C$ which does not contain $A$. The perpendicular bisectors of $AB$ and $AC$ meet the line $AU$ at $V$ and $W$, respectively. The lines $BV$ and $CW$ meet at $T$. Show that $AU = TB + TC$.
\vskip 12pt
\noindent {\bf A3}. Let $x_1, x_2, \ldots , x_n$ be real numbers satisfying $|x_1 + x_2 + \cdots + x_n| = 1$ and $|x_i| \le (n+1)/2$ for all $i$. Show that there exists a permutation $y_i$ of $x_i$ such that $|y_1 + 2 y_2 + \cdots + n y_n| \le (n+1)/2$.
\vskip 12pt
\noindent {\bf B1}. An $n \times n$ matrix whose entries come from the set $S = \{1, 2, \ldots , 2n-1\}$ is called a silver matrix if, for each $i = 1, 2, \ldots , n$, the $i\/$th row and the $i\/$th column together contain all elements of $S$. Show that:

(a) there is no silver matrix for $n = 1997$;

(b) silver matrices exist for infinitely many values of $n$.
\vskip 12pt
\noindent {\bf B2}. Find all pairs $(a,b)$ of positive integers that satisfy $a^{b^2} = b^a$.
\vskip 12pt
\noindent {\bf B3}. For each positive integer $n$, let $f(n)$ denote the number of ways of representing $n$ as a sum of powers of $2$ with non-negative integer exponents. Representations which differ only in the ordering of their summands are considered to be the same. For example, $f(4) = 4$, because $4$ can be represented as $4, 2+2, 2+1+1$ or $1+1+1+1$. Prove that for any integer $n \ge 3, 2^{n^2/4} < f(2^n) < 2^{n^2/2}$.
\vskip 40pt
\noindent {\bf 39th IMO 1998}
\vskip 25pt
\noindent {\bf A1}. In the convex quadrilateral $ABCD$, the diagonals $AC$ and $BD$ are perpendicular and the opposite sides $AB$ and $DC$ are not parallel. The point $P$, where the perpendicular bisectors of $AB$ and $CD$ meet, is inside $ABCD$. Prove that $ABCD$ is cyclic iff the triangles $ABP$ and $CDP$ have equal areas.
\vskip 12pt
\noindent {\bf A2}. In a competition there are $a$ contestants and $b$ judges, where $b \ge 3$ is an odd integer. Each judge rates each contestant as either ``pass" or ``fail". Suppose k is a number such that for any two judges their ratings coincide for at most k contestants. Prove $k/a \ge (b-1)/2b$.
\vskip 12pt
\noindent {\bf A3}.  For any positive integer $n$, let $d(n)$ denote the number of positive divisors of $n$ (including $1$ and $n$). Determine all positive integers $k$ such that $d(n^2) = k d(n)$ for some $n$.
\vskip 12pt
\noindent {\bf B1}. Determine all pairs $(a, b)$ of positive integers such that $ab^2 + b + 7$ divides $a^2b+a+b$.
\vskip 12pt
\noindent {\bf B2}. Let $I$ be the incenter of the triangle $ABC$. Let the incircle of $ABC$ touch the sides $BC, CA, AB$ at $K, L, M$ respectively. The line through $B$ parallel to $MK$ meets the lines $LM$ and $LK$ at $R$ and $S$ respectively. Prove that $\angle RIS$ is acute.
\vskip 12pt
\noindent {\bf B3}. Consider all functions $f: N \to N$ on the positive integers satisfying $f(t^2f(s)) = sf(t)^2$ for all $s$ and $t$. Determine the least possible value of $f(1998)$.
\vskip 40pt
\noindent {\bf 40th IMO 1999}
\vskip 25pt
\noindent {\bf A1}. Find all finite sets $S$ of at least three points in the plane such that for all distinct points $A, B$ in $S$, the perpendicular bisector of $AB$ is an axis of symmetry for S.
\vskip 12pt
\noindent {\bf A2}. Let $n \ge 2$ be a fixed integer. Find the smallest constant $C$ such that for all non-negative reals $x_1, \ldots , x_n$: $$\sum_{i<j} x_i x_j (x_i^2 + x_j^2) \le C (\sum_i x_i)^4.$$ Determine when equality occurs.
\vskip 12pt
\noindent {\bf A3}. Given an $n \times n$ square board, with $n$ even. Two distinct squares of the board are said to be adjacent if they share a common side, but a square is not adjacent to itself. Find the minimum number of squares that can be marked so that every square (marked or not) is adjacent to at least one marked square. 
\vskip 12pt
\noindent {\bf B1}. Find all pairs $(n, p)$ of positive integers, such that: $p$ is prime; $n \le 2p$; and $(p-1)^n + 1$ is divisible by $n^{p-1}$.
\vskip 12pt
\noindent {\bf B2}. The circles $C_1$ and $C_2$ lie inside the circle $C$, and are tangent to it at $M$ and $N$, respectively. $C_1$ passes through the center of $C_2$. The common chord of $C_1$ and $C_2$, when extended, meets $C$ at $A$ and $B$. The lines $MA$ and $MB$ meet $C_1$ again at $E$ and $F$. Prove that the line $EF$ is tangent to $C_2$.
\vskip 12pt
\noindent {\bf B3}. Determine all functions $f: R \to R$ such that $f(x - f(y)) = f(f(y)) + x f(y) + f(x) - 1$ for all $x, y$ in $R$. [$R$ is the reals.]
\vskip 40pt
\noindent {\bf 41st IMO 2000}
\vskip 25pt
\noindent {\bf A1}. $AB$ is tangent to the circles $CAMN$ and $NMBD$. $M$ lies between $C$ and $D$ on the line $CD$, and $CD$ is parallel to $AB$. The chords $NA$ and $CM$ meet at $P$; the chords $NB$ and $MD$ meet at $Q$. The rays $CA$ and $DB$ meet at $E$. Prove that $PE = QE$.
\vskip 12pt
\noindent {\bf A2}. $A, B, C$ are positive reals with product $1$. Prove that $(A - 1 + {1 \over B})(B - 1 + {1 \over C})(C - 1 + {1 \over A}) \le 1$.
\vskip 12pt
\noindent {\bf A3}. $k$ is a positive real. $N$ is an integer greater than $1$. $N$ points are placed on a line, not all coincident. A {\it move} is carried out as follows. Pick any two points $A$ and $B$ which are not coincident. Suppose that $A$ lies to the right of $B$. Replace $B$ by another point $B'$ to the right of $A$ such that $AB' = k BA$. For what values of $k$ can we move the points arbitrarily far to the right by repeated moves?
\vskip 12pt
\noindent {\bf B1}. $100$ cards are numbered $1$ to $100$ (each card different) and placed in $3$ boxes (at least one card in each box). How many ways can this be done so that if two boxes are selected and a card is taken from each, then the knowledge of their sum alone is always sufficient to identify the third box?
\vskip 12pt
\noindent {\bf B2}. Can we find $N$ divisible by just $2000$ different primes, so that $N$ divides $2^N + 1$? [$N$ may be divisible by a prime power.]
\vskip 12pt
\noindent {\bf B3}. $A_1A_2A_3$ is an acute-angled triangle. The foot of the altitude from $A_i$ is $K_i$ and the incircle touches the side opposite $A_i$ at $L_i$. The line $K_1K_2$ is reflected in the line $L_1L_2$. Similarly, the line $K_2K_3$ is reflected in $L_2L_3$ and $K_3K_1$ is reflected in $L_3L_1$. Show that the three new lines form a triangle with vertices on the incircle.
\vskip 40pt
\noindent {\bf 42nd IMO 2001}
\vskip 25pt
\noindent {\bf A1}. $ABC$ is acute-angled. $O$ is its circumcenter. $X$ is the foot of the perpendicular from $A$ to $BC. \ \angle C \ge \angle B + 30^o$. Prove that $\angle A + \angle COX < 90^o$.
\vskip 12pt
\noindent {\bf A2}. $a, b, c$ are positive reals. Prove that ${a \over \sqrt{a^2 + 8bc}} + {b \over \sqrt{b^2 + 8ca}} + {c\over \sqrt{c^2 + 8ab}} \ge 1$.
\vskip 12pt
\noindent {\bf A3}. Integers are placed in each of the $441$ cells of a $21 \times 21$ array. Each row and each column has at most $6$ different integers in it. Prove that some integer is in at least 3 rows and at least 3 columns.
\vskip 12pt
\noindent {\bf B1}. Let $n_1, n_2, \ldots , n_m$ be integers where $m$ is odd. Let $x = (x_1, \ldots , x_m)$ denote a permutation of the integers $1, 2, \ldots , m$. Let $f(x) = x_1 n_1 + x_2 n_2 + \cdots + x_m n_m$. Show that for some distinct permutations $a, b$ the difference $f(a) - f(b)$ is a multiple of $m!$.
\vskip 12pt
\noindent {\bf B2}. $ABC$ is a triangle. $X$ lies on $BC$ and $AX$ bisects $\angle A$. $Y$ lies on $CA$ and $BY$ bisects $\angle B. \ \angle A = 60^o. \ AB + BX = AY + YB$. Find all possible values for $\angle B$.
\vskip 12pt
\noindent {\bf B3}. $K > L > M > N$ are positive integers such that $KM + LN = (K + L - M + N)(-K + L + M + N)$. Prove that $KL + MN$ is composite.
\vskip 40pt
\noindent {\bf 43rd IMO 2002}
\vskip 25pt
\noindent {\bf A1}. $S$ is the set  of all $(h, k)$ with $h, k$ non-negative integers such that $h+k < n$. Each element of $S$ is colored red or blue, so that if  $(h, k)$ is red and $h' \le h, k' \le k$, then $(h', k')$ is also red. A type 1 subset of $S$ has $n$ blue elements with different first member and a type 2 subset of $S$ has $n$ blue elements with different second member. Show that there are the same number of type 1 and type 2 subsets. 
\vskip 12pt
\noindent {\bf A2}. $BC$ is a diameter of a circle center $O$. $A$ is any point on the circle with $\angle AOC > 60^o. \ EF$ is the chord which is the perpendicular bisector of $AO. \ D$ is the midpoint of the minor arc $AB$. The line through $O$ parallel to $AD$ meets $AC$ at $J$. Show that $J$ is the incenter of triangle $CEF$.
\vskip 12pt
\noindent {\bf A3}. Find all pairs of integers $m > 2, n > 2$ such that there are infinitely many positive integers $k$ for which $k^n + k^2 - 1$ divides $k^m + k-1$.
\vskip 12pt
\noindent {\bf B1}. The positive divisors of the integer $n > 1$ are $d_1 < d_2 < \ldots < d_k$, so that $d_1 = 1, d_k = n$. Let $d = d_1 d_2 + d_2 d_3 + \cdots + d_{k-1}d_k$. Show that $d < n^2$ and find all $n$ for which $d$ divides $n^2$.
\vskip 12pt
\noindent {\bf B2}. Find all real-valued functions on the reals such that $(f(x) + f(y)) ((f(u) + f(v)) = f(xu - yv) + f(xv + yu)$ for all $x, y, u, v$.
\vskip 12pt
\noindent {\bf B3}. $n > 2$ circles of radius 1 are drawn in the plane so that no line meets more than two of the circles. Their centers are $O_1, O_2, \cdots , O_n$. Show that $\sum_{i<j} 1/O_iO_j \le (n-1)\pi /4$. 
\vskip 40pt
\noindent {\bf 44th IMO 2003}
\vskip 25pt
\noindent {\bf A1}. $S$ is the set $\{1, 2, 3, \ldots , 1000000\}$. Show that for any subset $A$ of $S$ with 101 elements we can find 100 distinct elements $x_i$ of $S$, such that the sets $\{a+x_i | a \in A\}$ are all pairwise disjoint. 
\vskip 12pt
\noindent {\bf A2}. Find all pairs $(m, n)$ of positive integers such that ${m^2 \over 2mn^2 - n^3 + 1}$ is a positive integer.
\vskip 12pt
\noindent {\bf A3}. A convex hexagon has the property that for any pair of opposite sides the distance between their midpoints is $\sqrt{3}/2$ times the sum of their lengths Show that all the hexagon\rq s angles are equal.
\vskip 12pt
\noindent {\bf B1}. $ABCD$ is cyclic. The feet of the perpendicular from $D$ to the lines $AB, BC, CA$ are $P, Q, R$ respectively. Show that the angle bisectors of $ABC$ and $CDA$ meet on the line $AC$ iff $RP = RQ$.
\vskip 12pt
\noindent {\bf B2}. Given $n > 2$ and reals $x_1 \le x_2 \le \cdots \le x_n$, show that $(\sum_{i,j} |x_i - x_j|)^2 \le {2 \over 3}(n^2 - 1) \sum_{i,j} (x_i - x_j)^2$. Show that we have equality iff the sequence is an arithmetic progression.
\vskip 12pt
\noindent {\bf B3}. Show that for each prime $p$, there exists a prime $q$ such that $n^p - p$ is not divisible by $q$ for any positive integer $n$.
\vskip 40pt

\noindent \copyright John Scholes

\noindent jscholes@kalva.demon.co.uk

\noindent 25 August 2003

\bye
