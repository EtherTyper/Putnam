\nopagenumbers
\noindent {\bf 30th IMO 1989}
\vskip 25pt
\noindent {\bf A1}. Prove that the set $\{1,2,\ldots,1989\}$ can be expressed as the disjoint union of subsets $A_1,A_2,\ldots,A_{117}$ in such a way that each $A_i$ contains $17$ elements and the sum of the elements in each $A_i$ is the same.
\vskip 12pt
\noindent {\bf A2}. In an acute-angled triangle $ABC$, the internal bisector of angle $A$ meets the circumcircle again at $A_1$. Points $B_1$ and $C_1$ are defined similarly. Let $A_0$ be the point of intersection of the line $AA_1$ with the external bisectors of angles $B$ and $C$. Points $B_0$ and $C_0$ are defined similarly. Prove that the area of the triangle $A_0B_0C_0$ is twice the area of the hexagon $AC_1BA_1CB_1$ and at least four times the area of the triangle $ABC$.
\vskip 12pt
\noindent {\bf A3}. Let $n$ and $k$ be positive integers, and let $S$ be a set of $n$ points in the plane such that no three points of $S$ are collinear, and for any points $P$ of $S$ there are at least $k$ points of $S$ equidistant from $P$. Prove that $k<{1\over2}+\sqrt{2n}$.
\vskip 12pt
\noindent {\bf B1}. Let $ABCD$ be a convex quadrilateral such that the sides $AB,AD,BC$ satisfy $AB=AD+BC$. There exists a point $P$ inside the quadrilateral at a distance $h$ from the line $CD$ such that $AP=h+AD$ and $BP=h+BC$. Show that: $${1\over\sqrt h}\ge{1\over\sqrt{AD}}+{1\over\sqrt{BC}}.$$
\vskip 12pt
\noindent {\bf B2}. Prove that for each positive integer $n$ there exist $n$ consecutive positive integers none of which is a prime or a prime power.
\vskip 12pt
\noindent {\bf B3}. A permutation $\{x_1,x_2,\ldots,x_m\}$ of the set $\{1,2,\ldots,2n\}$ where $n$ is a positive integer is said to have property $P$ if $|x_i-x_{i+1}|=n$ for at least one $i$ in $\{1,2,\ldots,2n-1\}$. Show that for each $n$ there are more permutations with property $P$ than without.
\vskip 20pt
\noindent \copyright John Scholes

\noindent jscholes@kalva.demon.co.uk

\noindent 19 August 2003

\bye
