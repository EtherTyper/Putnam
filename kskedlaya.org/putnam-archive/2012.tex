\documentclass[amssymb,twocolumn,pra,10pt,aps]{revtex4-1}
\usepackage{mathptmx,amsmath}

\begin{document}
\title{The 73rd William Lowell Putnam Mathematical Competition \\
    Saturday, December 1, 2012}
\maketitle

\newcommand{\FF}{\mathbb{F}}
\newcommand{\RR}{\mathbb{R}}

\begin{itemize}

\item[A1] Let $d_1, d_2, \dots, d_{12}$ be real numbers in the open
interval $(1, 12)$. Show that there exist distinct indices $i, j, k$
such that $d_i, d_j, d_k$ are the side lengths of an acute triangle.

\item[A2]
Let $*$ be a commutative and associative binary operation on a set $S$. Assume that for every $x$
and $y$ in $S$, there exists $z$ in $S$ such that $x * z = y$. (This $z$ may depend on $x$ and $y$.)
Show that if $a,b,c$ are in $S$ and $a*c = b*c$, then $a=b$.

\item[A3]
Let $f: [-1, 1] \to \RR$ be a continuous function such that
\begin{itemize}
\item[(i)]
$f(x) = \frac{2-x^2}{2} f \left( \frac{x^2}{2-x^2} \right)$ for every $x$ in $[-1, 1]$,
\item[(ii)]
$f(0) = 1$, and
\item[(iii)]
$\lim_{x \to 1^-} \frac{f(x)}{\sqrt{1-x}}$ exists and is finite.
\end{itemize}
Prove that $f$ is unique, and express $f(x)$ in closed form.

\item[A4]
Let $q$ and $r$ be integers with $q > 0$, and let $A$ and $B$ be intervals on the real line.
Let $T$ be the set of all $b+mq$ where $b$ and $m$ are integers with $b$ in $B$,
and let $S$ be the set of all integers $a$ in $A$ such that $ra$ is in $T$. Show that if the
product of the lengths of $A$ and $B$ is less than $q$, then $S$ is the intersection of $A$
with some arithmetic progression.

\item[A5]
Let $\FF_p$ denote the field of integers modulo a prime $p$, and let $n$ be a positive integer.
Let $v$ be a fixed vector in $\FF_p^n$, let $M$ be an $n \times n$ matrix with entries of $\FF_p$,
and define $G: \FF_p^n \to \FF_p^n$ by $G(x) = v + Mx$. Let $G^{(k)}$ denote the $k$-fold
composition of $G$ with itself, that is, $G^{(1)}(x) = G(x)$ and $G^{(k+1)}(x) = G(G^{(k)}(x))$.
Determine all pairs $p, n$ for which there exist $v$ and $M$ such that the $p^n$ vectors
$G^{(k)}(0)$, $k=1,2,\dots,p^n$ are distinct.

\item[A6]
Let $f(x,y)$ be a continuous, real-valued function on $\RR^2$. Suppose that, for every
rectangular region $R$ of area $1$, the double integral of $f(x,y)$ over $R$ equals $0$.
Must $f(x,y)$ be identically 0?

\item[B1]
Let $S$ be a class of functions from $[0, \infty)$ to $[0, \infty)$ that satisfies:
\begin{itemize}
\item[(i)]
The functions $f_1(x) = e^x - 1$ and $f_2(x) = \ln(x+1)$ are in $S$;
\item[(ii)]
If $f(x)$ and $g(x)$ are in $S$, the functions $f(x) + g(x)$ and $f(g(x))$ are in $S$;
\item[(iii)]
If $f(x)$ and $g(x)$ are in $S$ and $f(x) \geq g(x)$ for all $x \geq 0$, then the function
$f(x) - g(x)$ is in $S$.
\end{itemize}
Prove that if $f(x)$ and $g(x)$ are in $S$, then the function $f(x) g(x)$ is also in $S$.

\item[B2]
Let $P$ be a given (non-degenerate) polyhedron. Prove that there is a constant $c(P) > 0$
with the following property: If a collection of $n$ balls whose volumes sum to $V$ contains
the entire surface of $P$, then $n > c(P) / V^2$.

\item[B3]
A round-robin tournament of $2n$ teams lasted for $2n-1$ days, as follows.
On each day, every team played one game against another team, with one team winning
and one team losing in each of the $n$ games. Over the course of the tournament,
each team played every other team exactly once. Can one necessarily choose
one winning team from each day without choosing any team more than once?

\item[B4]
Suppose that $a_0 = 1$ and that $a_{n+1} = a_n + e^{-a_n}$ for $n=0,1,2,\dots$. Does $a_n - \log n$
have a finite limit as $n \to \infty$? (Here $\log n = \log_e n = \ln n$.)

\item[B5]
Prove that, for any two bounded functions $g_1, g_2: \RR \to [1, \infty)$,
there exist functions $h_1, h_2: \RR \to \RR$ such that, for every $x \in \RR$,
\[
\sup_{s \in \RR} (g_1(s)^x g_2(s))  = \max_{t \in \RR} (x h_1(t) + h_2(t)).
\]

\item[B6]
Let $p$ be an odd prime number such that $p \equiv 2 \pmod{3}$. Define a permutation $\pi$ of the
residue classes modulo $p$ by $\pi(x) \equiv x^3 \pmod{p}$. Show that $\pi$ is an even permutation
if and only if $p \equiv 3 \pmod{4}$.

\end{itemize}

\end{document}
