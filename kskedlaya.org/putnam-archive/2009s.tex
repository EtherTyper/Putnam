\documentclass[amssymb,twocolumn,pra,10pt,aps]{revtex4-1}
\usepackage{mathptmx,amsmath}

\newcommand{\RR}{\mathbb{R}}
\newcommand{\CC}{\mathbb{C}}
\newcommand{\ZZ}{\mathbb{Z}}
\DeclareMathOperator{\lcm}{lcm}
\DeclareMathOperator{\sgn}{sgn}
\DeclareMathOperator{\Trace}{Trace}

\begin{document}
\title{Solutions to the 70th William Lowell Putnam Mathematical Competition \\
    Saturday, December 5, 2009}
\author{Kiran Kedlaya and Lenny Ng}
\noaffiliation
\maketitle

\begin{itemize}

\item[A--1]
Yes, it does follow. Let $P$ be any point in the plane. Let $ABCD$ be any square with center $P$.
Let $E,F,G,H$ be the midpoints of the segments $AB, BC, CD, DA$, respectively. The function
$f$ must satisfy the equations
\begin{align*}
0 &= f(A) + f(B) + f(C) + f(D) \\
0 &= f(E) + f(F) + f(G) + f(H) \\
0 &= f(A) + f(E) + f(P) + f(H) \\
0 &= f(B) + f(F) + f(P) + f(E) \\
0 &= f(C) + f(G) + f(P) + f(F) \\
0 &= f(D) + f(H) + f(P) + f(G).
\end{align*}
If we add the last four equations, then subtract the first equation and twice the second equation,
we obtain $0 = 4f(P)$, whence $f(P) = 0$.

\textbf{Remark.} Problem 1 of the 1996 Romanian IMO team selection exam asks the same
question with squares replaced by regular polygons of any (fixed) number of vertices.

\item[A--2]
Multiplying the first differential equation by $gh$, the second by $fh$,
and the third by $fg$, and summing gives
\[
(fgh)' = 6(fgh)^2+6.
\]
Write $k(x) = f(x)g(x)h(x)$; then $k' = 6k^2+6$ and $k(0) = 1$. One
solution for this differential equation with this initial condition is
$k(x) = \tan(6x+\pi/4)$; by standard uniqueness, this must necessarily
hold for $x$ in some open interval around $0$. Now the first given
equation becomes
\begin{align*}
f'/f &= 2k(x)+1/k(x) \\
&= 2\tan(6x+\pi/4)+\cot(6x+\pi/4);
\end{align*}
integrating both sides gives
\[
\ln(f(x)) = \frac{-2\ln\cos(6x+\pi/4) + \ln\sin(6x+\pi/4)}{6}+c,
\]
whence $f(x) = e^c
\left(\frac{\sin(6x+\pi/4)}{\cos^2(6x+\pi/4)}\right)^{1/6}$.
Substituting $f(0)=1$ gives $e^c = 2^{-1/12}$ and thus $f(x) = 2^{-1/12}
\left(\frac{\sin(6x+\pi/4)}{\cos^2(6x+\pi/4)}\right)^{1/6}$.

\textbf{Remark.} The answer can be put in alternate forms using
trigonometric identities. One particularly simple one is
\[
 f(x) = (\sec 12x)^{1/12} (\sec 12x + \tan 12x)^{1/4}.
\]

\item[A--3]
The limit is $0$; we will show this by checking that $d_n = 0$ for all $n \geq 3$.
Starting from the given matrix, add the third column to the first column; this does not change the
determinant. However, thanks to the identity
$\cos x + \cos y = 2 \cos \frac{x+y}{2} \cos \frac{x-y}{2}$,
the resulting matrix has the form
\[
 \begin{pmatrix} 2 \cos 2 \cos 1 & \cos 2 & \cdots \\
  2 \cos (n+2) \cos 1 & \cos (n+2) & \cdots \\
  2 \cos (2n+2) \cos 1 & 2 \cos (2n+2) & \cdots \\
\vdots & \vdots & \ddots
 \end{pmatrix}
\]
with the first column being a multiple of the second. Hence $d_n = 0$.

\textbf{Remark.}
Another way to draw the same conclusion is to observe that the given matrix is the sum
of the two rank 1 matrices $A_{jk} = \cos (j-1)n \cos k$ and $B_{jk} = -\sin (j-1)n \sin k$,
and so has rank at most 2. One can also use the matrices
$A_{jk} = e^{i((j-1)n+k)}$, $B_{jk} = e^{-i(j-1)n+k}$.

\item[A--4]
The answer is no; indeed, $S = \mathbb{Q} \setminus \{n+2/5 \,|\,
n\in\mathbb{Z}\}$ satisfies the given conditions. Clearly $S$ satisfies
(a) and (b); we need only check that it satisfies (c). It suffices to
show that if $x = p/q$ is a fraction with $(p,q)=1$ and $p>0$, then we
cannot have $1/(x(x-1)) = n+2/5$ for an integer $n$. Suppose otherwise; then
\[
(5n+2)p(p-q) = 5q^2.
\]
Since $p$ and $q$ are relatively prime, and $p$ divides $5q^2$, we must
have $p\,|\,5$, so $p=1$ or $p=5$. On the other hand, $p-q$ and $q$ are
also relatively prime, so $p-q$ divides $5$ as well, and $p-q$ must be
$\pm 1$ or $\pm 5$. This leads to eight possibilities for $(p,q)$:
$(1,0)$,  $(5,0)$,  $(5,10)$, $(1,-4)$, $(1,2)$, $(1,6)$, $(5,4)$,
$(5,6)$. The first three are impossible, while the final five lead to
$5n+2 = 16,-20,-36,16,-36$ respectively, none of which holds for
integral $n$.

\textbf{Remark.} More generally, no rational number of the form $m/n$,
where $m,n$ are relatively prime and neither of $\pm m$ is a quadratic
residue mod $n$, need be in $S$. If $x=p/q$ is in lowest terms and
$1/(x(x-1)) = m/n+k$ for some integer $k$, then $p(p-q)$ is relatively
prime to $q^2$; $q^2/(p(p-q)) = (m+kn)/n$ then implies that $m+kn = \pm
q^2$ and so $\pm m$ must be a quadratic residue mod $n$.

\item[A--5]
No, there is no such group.
By the structure theorem for finitely generated abelian groups,
$G$ can be written as a product of cyclic groups.
If any of these factors has odd order, then $G$ has an element of odd order,
so the product of the orders of all of its elements cannot be a power of 2.

We may thus consider only abelian $2$-groups hereafter.
For such a group $G$, the product of the orders of all of its elements
has the form $2^{k(G)}$ for some nonnegative integer $G$, and we must show
that it is impossible to achieve $k(G) = 2009$.
Again by the structure theorem, we may write
\[
G \cong \prod_{i=1}^\infty (\ZZ/2^i \ZZ)^{e_i}
\]
for some nonnegative integers $e_1,e_2,\dots$, all but finitely many of
which are $0$.

For any nonnegative integer $m$, the elements of $G$ of order at most $2^m$
form a subgroup isomorphic to
\[
\prod_{i=1}^\infty (\ZZ/2^{\min\{i,m\}} \ZZ)^{e_i},
\]
which has $2^{s_m}$ elements for $s_m = \sum_{i=1}^\infty \min\{i,m\} e_i$.
Hence
\[
k(G) = \sum_{i=1}^\infty i(2^{s_i} - 2^{s_{i-1}}).
\]
Since $s_1 \leq s_2 \leq \cdots$, $k(G)+1$ is always divisible by $2^{s_1}$.
In particular, $k(G) = 2009$ forces $s_1 \leq 1$.

However, the only cases where $s_1 \leq 1$ are where all of the $e_i$ are $0$,
in which case $k(G) = 0$, or where $e_i = 1$ for some $i$ and $e_j = 0$
for $j \neq i$, in which case $k(G) = (i-1)2^i + 1$.
The right side is a strictly increasing function
of $i$ which equals $1793$ for $i=8$ and $4097$ for $i=9$, so it can never equal
2009. This proves the claim.

\textbf{Remark.} One can also arrive at the key congruence by dividing $G$
into equivalence classes, by declaring two elements to be equivalent if they generate the
same cyclic subgroup of $G$.
For $h>0$, an element of order $2^h$ belongs to an equivalence class of size $2^{h-1}$,
so the products of the orders of the elements of this equivalence class is $2^j$
for $j = h 2^{h-1}$. This quantity is divisible by 4 as long as $h > 1$;
thus to have $k(G) \equiv 1 \pmod{4}$, the number of elements of $G$ of order 2 must be
congruent to 1 modulo 4. However, there are exactly $2^e-1$ such elements,
for $e$ the number of cyclic factors of $G$. Hence $e = 1$, and one concludes as in the
given solution.

\item[A--6]
We disprove the assertion using the example
\[
f(x,y) = 3(1 + y)(2x-1)^2 -y.
\]
We have $b-a = d-c = 0$ because the identity $f(x,y) = f(1-x,y)$ forces $a=b$,
and because
\begin{align*}
c &= \int_0^1 3(2x-1)^2\,dx = 1, \\
d &= \int_0^1 (6(2x-1)^2-1)\,dx = 1.
\end{align*}
Moreover, the partial derivatives
\begin{align*}
 \frac{\partial f}{\partial x}(x_0,y_0) &= 3(1+y_0)(8x_0-4) \\
\frac{\partial f}{\partial y}(x_0,y_0) &= 3(2x_0-1)^2-1.
\end{align*}
have no common zero in $(0,1)^2$. Namely,
for the first partial to vanish, we must have $x_0 = 1/2$ since
$1 + y_0$ is nowhere zero, but for $x_0 = 1/2$
the second partial cannot vanish.

\textbf{Remark.}
This problem amounts to refuting a potential generalization of the Mean Value Theorem to bivariate
functions. Many counterexamples are possible. Kent Merryfield suggests
$y \sin(2 \pi x)$, for which all four of the boundary integrals vanish;
here the partial derivatives are $2\pi y \cos (2 \pi x)$ and $\sin (2 \pi x)$.
Catalin Zara suggests $x^{1/3} y^{2/3}$. Qingchun Ren suggests $xy(1-y)$.

\item[B--1]
Every positive rational number can be uniquely written in lowest terms
as $a/b$ for $a,b$ positive integers. We prove the statement in the
problem by induction on the largest prime dividing either $a$ or $b$
(where this is considered to be $1$ if $a=b=1$). For the base case, we
can write $1/1 = 2!/2!$. For a general $a/b$, let $p$ be the largest
prime dividing either $a$ or $b$; then $a/b = p^k a'/b'$ for some $k\neq
0$ and positive integers $a',b'$ whose largest prime factors are
strictly less than $p$. We now have $a/b = (p!)^k \frac{a'}{(p-1)!^k b'}$,
and all prime factors of $a'$ and $(p-1)!^k b'$ are strictly less than $p$.
By the induction assumption, $\frac{a'}{(p-1)!^k b'}$ can be written as a
quotient of products of prime factorials, and so $a/b = (p!)^k
\frac{a'}{(p-1)!^k b'}$ can as well. This completes the induction.

\textbf{Remark.} Noam Elkies points out that the representations are unique
up to rearranging and canceling common factors.

\item[B--2]
The desired real numbers $c$ are precisely those for which $1/3 < c \leq 1$.
For any positive integer $m$ and any
sequence $0 = x_0 < x_1 < \cdots < x_m = 1$,
the cost of jumping along this sequence is
$\sum_{i=1}^m (x_i - x_{i-1})x_i^2$. Since
\begin{align*}
1 = \sum_{i=1}^m (x_i - x_{i-1}) &\geq \sum_{i=1}^m (x_i - x_{i-1})x_i^2 \\
&> \sum_{i=1}^m \int_{x_i}^{x_{i-1}} t^2\,dt \\
&= \int_0^1 t^2\,dt = \frac{1}{3},
\end{align*}
we can only achieve costs $c$ for which $1/3 < c \leq 1$.

It remains to check that any such $c$ can be achieved.
Suppose $0 = x_0 < \dots < x_m = 1$ is a sequence with $m \geq 1$.
For $i=1,\dots,m$,
let $c_i$ be the cost of the sequence $0, x_i, x_{i+1},\dots,x_m$.
For $i > 1$ and $0 < y \leq x_{i-1}$,
the cost of the sequence $0, y, x_{i}, \dots, x_m$
is
\[
c_{i} + y^3 + (x_i - y)x_i^2 - x_i^3
= c_i - y(x_i^2 - y^2),
\]
which is less than $c_i$ but approaches $c_i$ as $y \to 0$.
By continuity, for $i=2,\dots,m$,
every value in the interval $[c_{i-1}, c_{i})$ can be achieved,
as can $c_m = 1$ by the sequence $0,1$.

To show that all costs $c$ with $1/3 < c \leq 1$ can be achieved, it now suffices
to check that for every $\epsilon > 0$, there exists a sequence with cost at most
$1/3 + \epsilon$. For instance, if we take $x_i = i/m$ for $i=0,\dots,m$, the cost
becomes
\[
\frac{1}{m^3} (1^2 + \cdots + m^2)
 = \frac{(m+1)(2m+1)}{6m^2},
\]
which converges to $1/3$ as $m \to +\infty$.

\textbf{Reinterpretation.} The cost of jumping along a particular sequence is an
upper Riemann sum of the function $t^2$. The fact that this function admits a Riemann
integral implies that for any $\epsilon > 0$, there exists $\delta_0$ such that the
cost of the sequence $x_0,\dots,x_m$ is at most $1/3 + \epsilon$ as long as
$\max_i \{x_i - x_{i-1}\} < \epsilon$. (The computation of the integral using the
sequence $x_i = i/m$ was already known to Archimedes.)

\item[B--3]
The answer is $n=2^k-1$ for some integer $k\geq 1$.
There is a bijection between mediocre subsets of $\{1,\ldots,n\}$ and
mediocre subsets of $\{2,\ldots,n+1\}$ given by adding $1$ to each
element of the subset; thus $A(n+1)-A(n)$ is the number of mediocre
subsets of $\{1,\ldots,n+1\}$ that contain $1$. It follows that
$A(n+2)-2A(n+1)+A_n = (A(n+2)-A(n+1))-(A(n+1)-A(n))$ is the difference
between the number of mediocre subsets of $\{1,\ldots,n+2\}$ containing
$1$ and the number of mediocre subsets of $\{1,\ldots,n+1\}$ containing
$1$. This difference is precisely the number of mediocre subsets of
$\{1,\ldots,n+2\}$ containing both $1$ and $n+2$, which we term
``mediocre subsets containing the endpoints.'' Since $\{1,\ldots,n+2\}$
itself is a mediocre subset of itself containing the endpoints, it
suffices to prove that this is the only mediocre subset of
$\{1,\ldots,n+2\}$ containing the endpoints if and only if $n=2^k-1$ for
some $k$.

If $n$ is not of the form $2^k-1$, then we can write $n+1 = 2^a b$ for
odd $b>1$. In this case, the set $\{1+m b \, | \, 0 \leq m \leq 2^a\}$
is a mediocre subset of $\{1,\ldots,n+2\}$ containing the endpoints: the
average of $1+m_1 b$ and $1+m_2 b$, namely $1+\frac{m_1+m_2}{2} b$, is
an integer if and only if $m_1+m_2$ is even, in which case this average
lies in the set.

It remains to show that if $n=2^k-1$, then the only mediocre subset of
$\{1,\ldots,n+2\}$ containing the endpoints is itself. This is readily
seen by induction on $k$. For $k=1$, the statement is obvious. For
general $k$, any mediocre subset $S$ of $\{1,\ldots,n+2=2^k+1\}$
containing $1$ and $2^k+1$ must also contain their average, $2^{k-1}+1$.
By the induction assumption, the only mediocre subset of
$\{1,\ldots,2^{k-1}+1\}$ containing the endpoints is itself, and so $S$
must contain all integers between $1$ and $2^{k-1}+1$. Similarly, a
mediocre subset of $\{2^{k-1}+1,\ldots,2^k+1\}$ containing the endpoints
gives a mediocre subset of $\{1,\ldots,2^{k-1}+1\}$ containing the
endpoints by subtracting $2^{k-1}$ from each element. By the induction
assumption again, it follows that $S$ must contain all integers between
$2^{k-1}+1$ and $2^k+1$. Thus $S = \{1,\ldots,2^k+1\}$ and the induction
is complete.

\textbf{Remark.} One can also proceed by checking that a nonempty subset of
$\{1,\dots,n\}$ is mediocre if and only if it is an arithmetic progression
with odd common difference. Given this fact, the number of
mediocre subsets of $\{1,\dots,n+2\}$ containing the endpoints is seen to be
the number of odd factors of $n+1$, from which the desired result is evident.
(The sequence $A(n)$ appears as sequence A124197 in the Encyclopedia of Integer Sequences.)

\item[B--4]
Any polynomial $P(x,y)$ of degree at most $2009$ can be written uniquely
as a sum $\sum_{i=0}^{2009} P_i(x,y)$ in which $P_i(x,y)$ is a homogeneous
polynomial of degree $i$.
For $r>0$, let $C_r$ be the path $(r\cos \theta, r\sin \theta)$
for $0 \leq \theta \leq 2\pi$. Put $\lambda(P_i) = \oint_{C_1} P_i$; then
for $r>0$,
\[
\oint_{C_r} P = \sum_{i=0}^{2009} r^i \lambda(P_i).
\]
For fixed $P$, the right side is a polynomial in $r$, which vanishes for
all $r>0$ if and only if its coefficients vanish.
In other words,
$P$ is balanced
if and only if $\lambda(P_i) = 0$ for $i=0,\dots,2009$.

For $i$ odd, we have $P_i(-x,-y) = -P_i(x,y)$.
Hence $\lambda(P_i) = 0$, e.g.,
because the contributions to the integral from
$\theta$ and $\theta + \pi$ cancel.

For $i$ even, $\lambda(P_i)$ is a linear function of the coefficients of
$P_i$. This function is not identically zero, e.g., because for $P_i =
(x^2 + y^2)^{i/2}$, the integrand is always positive and so
$\lambda(P_i) > 0$. The kernel of $\lambda$ on the space of homogeneous
polynomials of degree $i$ is thus a subspace of codimension 1.

It follows that the dimension of $V$ is
\[
(1 + \cdots + 2010) - 1005 = (2011 - 1) \times 1005 = 2020050.
\]

\item[B--5]
\textbf{First solution.}
If $f(x) \geq x$ for all $x > 1$, then the desired conclusion clearly holds.
We may thus assume hereafter that there exists $x_0 > 1$ for which $f(x_0) < x_0$.

Rewrite the original differential equation as
\[
 f'(x) = 1 - \frac{x^2 + 1}{x^2} \frac{f(x)^2}{1 + f(x)^2}.
\]
Put $c_0 = \min\{0, f(x_0) - 1/x_0\}$.
For all $x \geq x_0$, we have $f'(x) > -1/x^2$ and so
\[
f(x) \geq f(x_0) -\int_{x_0}^x dt/t^2 > c_0.
\]
In the other direction, we claim that $f(x) < x$ for all $x \geq x_0$.
To see this, suppose the contrary; then by continuity, there is a least
$x \geq x_0$ for which $f(x) \geq x$, and this least value satisfies $f(x) = x$.
However, this forces $f'(x) = 0 < 1$ and so
$f(x-\epsilon) > x-\epsilon$ for $\epsilon > 0$ small,
contradicting the choice of $x$.

Put $x_1 = \max\{x_0, -c_0\}$. For $x \geq x_1$, we have
$|f(x)| < x$ and so $f'(x) > 0$.
In particular, the limit $\lim_{x \to +\infty} f(x) = L$
exists.

Suppose that $L < +\infty$; then $\lim_{x \to +\infty}
f'(x) = 1/(1 + L^2) > 0$. Hence for any sufficiently small $\epsilon > 0$,
we can choose $x_2 \geq x_1$ so that $f'(x) \geq \epsilon$
for $x \geq x_2$. But then $f(x) \geq f(x_2) + \epsilon(x-x_2)$,
which contradicts $L < +\infty$. Hence $L = +\infty$,
as desired.

\textbf{Variant.} (by Leonid Shteyman) One obtains a similar argument by
writing
\[
 f'(x) = \frac{1}{1 + f(x)^2} - \frac{f(x)^2}{x^2(1+f(x)^2)},
\]
so that
\[
-\frac{1}{x^2} \leq f'(x) - \frac{1}{1 + f(x)^2} \leq 0.
\]
Hence $f'(x) - 1/(1 + f(x)^2)$ tends to 0 as $x \to +\infty$, so $f(x)$ is bounded below, and
tends to $+\infty$
if and only if the improper integral $\int dx/(1+f(x)^2)$ diverges. However, if the integral were to
converge, then as $x \to +\infty$ we would have $1/(1+f(x)^2) \to 0$; however, since $f$ is bounded below,
this again forces $f(x) \to +\infty$.

\textbf{Second solution.} (by Catalin Zara)
The function $g(x) = f(x)+x$ satisfies the differential equation
\[
g'(x) = 1 + \frac{1 - (g(x)/x - 1)^2}{1 + x^2(g(x)/x - 1)^2}.
\]
This implies that $g'(x) > 0$ for all $x > 1$, so
the limit $L_1 = \lim_{x \to +\infty} g(x)$ exists. In addition, we cannot have
$L_1 < +\infty$, or else we would have
$\lim_{x \to +\infty} g'(x) = 0$ whereas the differential equation forces
this limit to be 1.
Hence $g(x) \to +\infty$ as $x \to +\infty$.

Similarly, the function $h(x) = -f(x) + x$ satisfies the differential equation
\[
h'(x) = 1 - \frac{1 - (h(x)/x - 1)^2}{1 + x^2(h(x)/x - 1)^2}.
\]
This implies that $h'(x) \geq 0$ for all $x$,
so the limit $L_2 = \lim_{x \to +\infty} h(x)$ exists. In addition, we cannot have
$L_2 < +\infty$, or else we would have
$\lim_{x \to +\infty} h'(x) = 0$ whereas the differential equation forces
this limit to be 1.
Hence $h(x) \to +\infty$ as $x \to +\infty$.

For some $x_1 > 1$, we must have $g(x), h(x) > 0$ for all $x \geq x_1$. For
$x \geq x_1$, we have $|f(x)| < x$ and hence $f'(x) > 0$, so the limit
$L = \lim_{x \to +\infty} f(x)$ exists. Once again,
we cannot have $L < +\infty$, or else we would have
$\lim_{x \to +\infty} f'(x) = 0$ whereas the original differential equation (e.g., in the form
given in the first solution) forces
this limit to be $1/(1 + L^2) > 0$.
Hence $f(x) \to +\infty$ as $x \to \infty$, as desired.

\textbf{Third solution.}
(by Noam Elkies)
Consider the function $g(x) = f(x) + \frac{1}{3}f(x)^3$, for which
\[
 g'(x) = f'(x)(1 + f(x)^2) = 1 - \frac{f(x)^2}{x^2}
\]
for $x>1$. Since evidently $g'(x) < 1$,
$g(x) - x$ is bounded above for $x$ large.
As in the first solution,
$f(x)$ is bounded below for $x$ large,
so $\frac{1}{3} f(x)^3 - x$ is bounded above by some $c>0$. For $x \geq c$,
we obtain $f(x) \leq (6x)^{1/3}$.

Since $f(x)/x \to 0$ as $x \to +\infty$, $g'(x) \to 1$ and so
$g(x)/x \to 1$. Since $g(x)$ tends to $+\infty$,
so does $f(x)$. (With a tiny bit of extra work, one shows that in fact $f(x)/(3x)^{1/3} \to 1$ as $x \to +\infty$.)

\item[B--6]
\textbf{First solution.}
(based on work of Yufei Zhao)
Since any sequence of the desired form remains of the desired form upon multiplying each term by 2,
we may reduce to the case where $n$ is odd. In this case, take $x = 2^h$ for some
positive integer $h$ for which $x \geq n$, and set
\begin{align*}
a_0 &= 0\\
a_1 &= 1\\
a_2 &= 2x+1 = a_1 + 2x \\
a_3 &= (x+1)^2 = a_2 + x^2 \\
a_4 &= x^n+1 = a_1 + x^n\\
a_5 &= n(x+1) = a_4 \mod a_3\\
a_6 &= x \\
a_7 &= n = a_5 \mod a_6.
\end{align*}
We may pad the sequence to the desired length by taking
$a_8 = \cdots = a_{2009} = n$.

\textbf{Second solution.}
(by James Merryfield)
Suppose first that $n$ is not divisible by 3. Recall that since $2$ is a primitive root modulo
$3^2$, it is also a primitive root modulo $3^h$ for any positive integer $h$. In particular,
if we choose $h$ so that $3^{2h} > n$, then
there exists a positive integer $c$ for which $2^c \mod 3^{2h} = n$.
We now take $b$ to be a positive integer for which $2^b > 3^{2h}$, and then put
\begin{align*}
a_0 &= 0\\
a_1 &= 1\\
a_2 &= 3 = a_1 + 2\\
a_3 &= 3 + 2^b \\
a_4 &= 2^{2hb} \\
a_5 &= 3^{2h} = a_4 \mod a_3 \\
a_6 &= 2^c \\
a_7 &= n = a_6 \mod a_5.
\end{align*}
If $n$ is divisible by 3, we can force $a_7 = n-1$ as in the above
construction, then put $a_{8} = a_7 + 1 = n$. In both cases, we then pad the sequence
as in the first solution.

\textbf{Remark.}
Hendrik Lenstra, Ronald van Luijk, and Gabriele Della Torre
suggest the following variant of the first solution requiring only 6 steps.
For $n$ odd and $x$ as in the first solution, set
\begin{align*}
a_0 &= 0\\
a_1 &= 1\\
a_2 &= x+1 = a_1 + x\\
a_3 &= x^n+x+1 = a_2 + x^n\\
a_4 &= x^{(n-1)(\phi(a_3)-1)}\\
a_5 &= \frac{x^n+1}{x+1} = a_4 \mod a_3 \\
a_6 &= n = a_5 \mod a_2.
\end{align*}
It seems unlikely that a shorter solution can be constructed without relying on
any deep number-theoretic conjectures.

\end{itemize}
\end{document}



