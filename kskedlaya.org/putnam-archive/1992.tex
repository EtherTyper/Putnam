\documentclass[amssymb,twocolumn,pra,10pt,aps]{revtex4-1}
\usepackage{mathptmx,amsmath}

\begin{document}
\title{The 53rd William Lowell Putnam Mathematical Competition \\
    Saturday, December 5, 1992}
\maketitle

\newcommand{\MM}{{\mathcal{M}}}

\begin{itemize}
\item[A--1] Prove that $f(n) = 1-n$ is the only integer-valued function
defined on the integers that satisfies the following conditions.
\begin{itemize}
\item[(i)] $f(f(n)) = n$, for all integers $n$;
\item[(ii)] $f(f(n+2)+2) = n$ for all integers $n$;
\item[(iii)] $f(0) = 1$.
\end{itemize}

\item[A--2] Define $C(\alpha)$ to be the coefficient of $x^{1992}$ in the
power series about $x=0$ of $(1 + x)^\alpha$. Evaluate
\[
\int_0^1 \left( C(-y-1) \sum_{k=1}^{1992} \frac{1}{y+k} \right)\,dy.
\]

\item[A--3] For a given positive integer $m$, find all triples $(n, x, y)$
of positive integers, with $n$ relatively prime to $m$, which satisfy
\[
(x^2 + y^2)^m = (xy)^n.
\]

\item[A--4] Let $f$ be an infinitely differentiable real-valued function
defined on the real numbers. If
\[
f\left( \frac{1}{n} \right) = \frac{n^2}{n^2 + 1}, \qquad n = 1, 2, 3, \dots,
\]
compute the values of the derivatives $f^{(k)}(0), k = 1, 2, 3, \dots$.

\item[A--5] For each positive integer $n$, let $a_n = 0$ (or 1) if the
number of 1's in the binary representation of $n$ is even (or odd),
respectively. Show that there do not exist positive integers $k$ and $m$
such that
\[
a_{k+j} = a_{k+m+j} = a_{k+2m+j},
\]
for $0 \leq j \leq m-1$.

\item[A--6] Four points are chosen at random on the surface of a sphere.
What is the probability that the center of the sphere lies inside the
tetrahedron whose vertices are at the four points? (It is understood that
each point is independently chosen relative to a uniform distribution on
the sphere.)

\item[B--1] Let $S$ be a set of $n$ distinct real numbers. Let $A_S$ be
the set of numbers that occur as averages of two distinct elements of
$S$. For a given $n \geq 2$, what is the smallest possible number of
elements in $A_S$?

\item[B--2] For nonnegative integers $n$ and $k$, define $Q(n, k)$ to be
the coefficient of $x^k$ in the expansion of $(1 + x + x^2 + x^3)^n$.
Prove that
\[
Q(n, k) = \sum_{j=0}^k \binom{n}{j} \binom{n}{k-2j},
\]
where $\binom{a}{b}$ is the standard binomial coefficient. (Reminder: For
integers $a$ and $b$ with $a \geq 0$, $\binom{a}{b} =
\frac{a!}{b!(a-b)!}$ for $0 \leq b \leq a$, with $\binom{a}{b} = 0$ otherwise.)

\item[B--3] For any pair $(x, y)$ of real numbers, a sequence
$(a_n(x,y))_{n\geq 0}$ is defined as follows:
\begin{align*}
a_0(x, y) &= x, \\
a_{n+1}(x, y) &= \frac{(a_n(x, y))^2 + y^2}{2}, \qquad \mbox{for $n \geq 0$}.
\end{align*}
Find the area of the region
\[
\{ (x, y) | (a_n(x, y))_{n \geq 0}\ \mbox{converges}\}.
\]

\item[B--4] Let $p(x)$ be a nonzero polynomial of degree less than 1992
having no nonconstant factor in common with $x^3 - x$. Let
\[
\frac{d^{1992}}{dx^{1992}} \left( \frac{p(x)}{x^3 - x} \right) =
\frac{f(x)}{g(x)}
\]
for polynomials $f(x)$ and $g(x)$. Find the smallest possible degree of $f(x)$.

\item[B--5] Let $D_n$ denote the value of the $(n-1) \times (n-1)$ determinant
\[
\left[
\begin{array}{cccccc}
3 & 1 & 1 & 1 & \cdots & 1 \\
1 & 4 & 1 & 1 & \cdots & 1 \\
1 & 1 & 5 & 1 & \cdots & 1 \\
1 & 1 & 1 & 6 & \cdots & 1 \\
\vdots & \vdots & \vdots & \vdots & \ddots & \vdots \\
1 & 1 & 1 & 1 & \cdots & n+1
\end{array}
\right].
\]
Is the set $\left\{ \frac{D_n}{n!} \right\}_{n \geq 2}$ bounded?

\item[B--6] Let $\MM$ be a set of real $n \times n$ matrices such that
\begin{itemize}
\item[(i)] $I \in \MM$, where $I$ is the $n \times n$ identity matrix;

\item[(ii)] if $A \in \MM$ and $B \in \MM$, then either $AB \in \MM$ or
$-AB \in \MM$, but not both;

\item[(iii)] if $A \in \MM$ and $B \in \MM$, then either $AB = BA$ or
$AB = -BA$;

\item[(iv)] if $A \in \MM$ and $A \neq I$, there is at least one $B \in
\MM$ such that $AB = - BA$.
\end{itemize}
Prove that $\MM$ contains at most $n^2$ matrices.
\end{itemize}
\end{document}
