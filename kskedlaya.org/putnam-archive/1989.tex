\documentclass[amssymb,twocolumn,pra,10pt,aps]{revtex4-1}
\usepackage{mathptmx,amsmath}

\begin{document}
\title{The Fiftieth Annual William Lowell Putnam Competition \\
    Saturday, December 2, 1989}
\maketitle

\begin{itemize}
\item[A--1]
How many primes among the positive integers, written as usual in base 10,
are alternating 1's and 0's, beginning and ending with 1?

\item[A--2]
Evaluate
$\displaystyle{\int_0^a\int_0^b e^{{\rm max}\{b^2x^2, a^2y^2\}}\,dy\,dx}$
where $a$ and $b$ are positive.

\item[A--3]
Prove that if
\[
11z^{10}+10iz^9+10iz-11=0,
\]
then $|z|=1.$ (Here $z$ is a complex number and $i^2=-1$.)

\item[A--4]
If $\alpha$ is an irrational number, $0 < \alpha < 1$, is there a
finite game with an honest coin such that the probability of one player
winning the game is $\alpha$? (An honest coin is one for which the
probability of heads and the probability of tails are both $\frac12$.
A game is finite if with probability 1 it must end in a finite number of moves.)

\item[A--5]
Let $m$ be a positive integer and let $\mathcal{G}$ be a regular $(2m+1)$-gon
inscribed in the unit circle. Show that there is a positive constant $A$,
independent of $m$, with the following property. For any points $p$ inside
$\cal G$ there are two distinct vertices $v_1$ and $v_2$ of $\cal G$
such that
\[
\left|\,|p-v_1| - |p-v_2|\,\right| < \frac1{m} - \frac{A}{m^3}.
\]
Here $|s-t|$ denotes the distance between the points $s$ and $t$.

\item[A--6]
Let $\alpha=1+a_1x+a_2x^2+\cdots$ be a formal power series with coefficients
in the field of two elements. Let
\[
a_n =
\begin{cases}
1 & \parbox{2in}{if every block of zeros in the binary expansion of $n$
has an even number of zeros in the block} \\[.3in]
0 & \text{otherwise.}
\end{cases}
\]
(For example, $a_{36}=1$ because $36=100100_2$ and $a_{20}=0$ because
$20=10100_2.$)
Prove that $\alpha^3+x\alpha+1=0.$

\item[B--1]
A dart, thrown at random, hits a square target. Assuming that any two
parts of the target of equal area are equally likely to be hit, find
the probability that the point hit is nearer to the center than to any
edge. Express your answer in the form $\displaystyle{\frac{a\sqrt{b} + c}{d}}$,
where $a,\,b,\,c,\,d$ are integers.

\item[B--2]
Let $S$ be a non-empty set with an associative operation that is left and
right cancellative ($xy=xz$ implies $y=z$, and $yx=zx$ implies $y=z$).
Assume that for every $a$ in $S$ the set $\{a^n:\,n=1, 2, 3, \ldots\}$ is
finite. Must $S$ be a group?

\item[B--3]
Let $f$ be a function on $[0,\infty)$, differentiable and satisfying
\[
f'(x)=-3f(x)+6f(2x)
\]
for $x>0$. Assume that $|f(x)|\le e^{-\sqrt{x}}$ for $x\ge 0$ (so that
$f(x)$ tends rapidly to $0$ as $x$ increases).
For $n$ a non-negative integer, define
\[
\mu_n=\int_0^\infty x^n f(x)\,dx
\]
(sometimes called the $n$th moment of $f$).
\begin{enumerate}
\item[a)] Express $\mu_n$ in terms of $\mu_0$.
\item[b)] Prove that the sequence $\{\mu_n \frac{3^n}{n!}\}$ always converges,
and that the limit is $0$ only if $\mu_0=0$.
\end{enumerate}

\item[B--4]
Can a countably infinite set have an uncountable collection of
non-empty subsets such that the intersection of any two of them is
finite?

\item[B--5]
Label the vertices of a trapezoid $T$ (quadrilateral with two parallel sides)
inscribed in the unit circle as $A,\,B,\,C,\,D$ so that $AB$ is parallel to
$CD$ and $A,\,B,\,C,\,D$ are in counterclockwise order. Let
$s_1,\,s_2$, and $d$ denote the lengths of the line segments
$AB,\, CD$, and $OE$, where E  is the point of intersection of the diagonals
of $T$, and $O$ is the center of the circle. Determine the least upper bound of
$\frac{s_1-s_2}{d}$ over all such $T$ for which $d\ne 0$, and describe all
cases, if any, in which it is attained.

\item[B--6]
Let $(x_1,\,x_2,\,\ldots\,x_n)$ be a point chosen at random from the
$n$-dimensional region defined by
$0<x_1<x_2<\cdots < x_n<1.$  Let $f$ be a continuous function on
$[0,1]$ with $f(1)=0$.
Set $x_0=0$ and $x_{n+1}=1$. Show that the expected value of the
Riemann sum
\[
\sum_{i=0}^n (x_{i+1}-x_i) f(x_{i+1})
\]
is $\int_0^1 f(t)P(t)\, dt$, where $P$ is a polynomial of degree $n$,
independent of $f$, with $0\le P(t)\le 1$ for $0\le t \le 1$.

\end{itemize}
\end{document}
